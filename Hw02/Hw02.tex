\documentclass{article}

\usepackage{fancyhdr}
\usepackage{extramarks}
\usepackage{amsmath}
\usepackage{amsthm}
\newtheorem{lemma}{Lemma}
\usepackage{amsfonts}
\usepackage{tikz}
\usepackage[plain]{algorithm}
\usepackage{algpseudocode}

\usetikzlibrary{automata,positioning}

%
% Basic Document Settings
%

\topmargin=-0.45in
\evensidemargin=0in
\oddsidemargin=0in
\textwidth=6.5in
\textheight=9.0in
\headsep=0.25in

\linespread{1.1}

\pagestyle{fancy}
\lhead{\hmwkAuthorName}
\chead{\hmwkClass\ (\hmwkClassInstructor\ \hmwkClassTime): \hmwkTitle}
\rhead{\firstxmark}
\lfoot{\lastxmark}
\cfoot{\thepage}

\renewcommand\headrulewidth{0.4pt}
\renewcommand\footrulewidth{0.4pt}

\setlength\parindent{0pt}

%
% Create Problem Sections
%

\newcommand{\enterProblemHeader}[1]{
    \nobreak\extramarks{}{Problem \arabic{#1} continued on next page\ldots}\nobreak{}
    \nobreak\extramarks{Problem \arabic{#1} (continued)}{Problem \arabic{#1} continued on next page\ldots}\nobreak{}
}

\newcommand{\exitProblemHeader}[1]{
    \nobreak\extramarks{Problem \arabic{#1} (continued)}{Problem \arabic{#1} continued on next page\ldots}\nobreak{}
    \stepcounter{#1}
    \nobreak\extramarks{Problem \arabic{#1}}{}\nobreak{}
}

\setcounter{secnumdepth}{0}
\newcounter{partCounter}
\newcounter{homeworkProblemCounter}
\setcounter{homeworkProblemCounter}{1}
\nobreak\extramarks{Problem \arabic{homeworkProblemCounter}}{}\nobreak{}

%
% Homework Problem Environment
%
% This environment takes an optional argument. When given, it will adjust the
% problem counter. This is useful for when the problems given for your
% assignment aren't sequential. See the last 3 problems of this template for an
% example.
%
\newenvironment{homeworkProblem}[1][-1]{
    \ifnum#1>0
        \setcounter{homeworkProblemCounter}{#1}
    \fi
    \section{Problem \arabic{homeworkProblemCounter}}
    \setcounter{partCounter}{1}
    \enterProblemHeader{homeworkProblemCounter}
}{
    \exitProblemHeader{homeworkProblemCounter}
}

%
% Homework Details
%   - Title
%   - Due date
%   - Class
%   - Section/Time
%   - Instructor
%   - Author
%

\newcommand{\hmwkTitle}{Homework\ \#2}
\newcommand{\hmwkDueDate}{Mar 12, 2024}
\newcommand{\hmwkClass}{Real Analysis}
\newcommand{\hmwkClassTime}{Tuesday}
\newcommand{\hmwkClassInstructor}{Professor Yakun Xi}
\newcommand{\hmwkAuthorName}{\textbf{Shuang Hu}}

%
% Title Page
%

\title{
    \vspace{2in}
    \textmd{\textbf{\hmwkClass:\ \hmwkTitle}}\\
    \normalsize\vspace{0.1in}\small{Due\ on\ \hmwkDueDate\ at 10:00am}\\
    \vspace{0.1in}\large{\textit{\hmwkClassInstructor\ \hmwkClassTime}}
    \vspace{3in}
}

\author{\hmwkAuthorName}
\date{}

\renewcommand{\part}[1]{\textbf{\large Part \Alph{partCounter}}\stepcounter{partCounter}\\}

%
% Various Helper Commands
%

% Useful for algorithms
\newcommand{\alg}[1]{\textsc{\bfseries \footnotesize #1}}

% For derivatives
\newcommand{\deriv}[1]{\frac{\mathrm{d}}{\mathrm{d}x} (#1)}

% For partial derivatives
\newcommand{\pderiv}[2]{\frac{\partial}{\partial #1} (#2)}

% Integral dx
\newcommand{\dx}{\mathrm{d}x}

% Alias for the Solution section header
\newcommand{\solution}{\textbf{\large Solution}}
\newcommand{\norm}[1]{\|#1\|}
% Probability commands: Expectation, Variance, Covariance, Bias
\newcommand{\Var}{\mathrm{Var}}
\newcommand{\Cov}{\mathrm{Cov}}
\newcommand{\Bias}{\mathrm{Bias}}
\newcommand{\supp}{\text{supp}}
\newcommand{\Rn}{\mathbb{R}^{n}}
\newcommand{\dif}{\mathrm{d}}
\newcommand{\avg}[1]{\left\langle #1 \right\rangle}
\newcommand{\difFrac}[2]{\frac{\dif #1}{\dif #2}}
\newcommand{\pdfFrac}[2]{\frac{\partial #1}{\partial #2}}
\newcommand{\OFL}{\mathrm{OFL}}
\newcommand{\UFL}{\mathrm{UFL}}
\newcommand{\fl}{\mathrm{fl}}
\newcommand{\Eabs}{E_{\mathrm{abs}}}
\newcommand{\Erel}{E_{\mathrm{rel}}}
\newcommand{\DR}{\mathcal{D}_{\widetilde{LN}}^{n}}
\newcommand{\add}[2]{\sum_{#1=1}^{#2}}
\newcommand{\innerprod}[2]{\left<#1,#2\right>}
\newcommand{\Sc}{\mathcal{S}}
\newcommand{\F}{\mathcal{F}}
\newcommand{\E}{\mathcal{E}}
\newcommand{\cp}[1]{\cup_{#1=1}^{\infty}}
\newcommand{\sm}[1]{\sum_{#1=1}^{\infty}}
\newcommand{\M}{\mathcal{M}}
\newcommand\tbbint{{-\mkern -16mu\int}}
\newcommand\tbint{{\mathchar '26\mkern -14mu\int}}
\newcommand\dbbint{{-\mkern -19mu\int}}
\newcommand\dbint{{\mathchar '26\mkern -18mu\int}}
\newcommand\bint{
{\mathchoice{\dbint}{\tbint}{\tbint}{\tbint}}
}
\newcommand\bbint{
{\mathchoice{\dbbint}{\tbbint}{\tbbint}{\tbbint}}
}
\begin{document}
\maketitle
\pagebreak
\begin{homeworkProblem}
    Complete the proof of Theorem 1.9.
\end{homeworkProblem}
\begin{proof}
    First, we show that $\bar{\mu}$ be a complete measure on 
    $\bar{\mathcal{M}}$. We divide this proof in 
    the following three steps.

    First, $\forall\mathcal{S}\in\bar{\mathcal{M}}$, 
    $\mathcal{S}$ can be divided to 
    $\mathcal{S}=\mathcal{E}\cup\mathcal{F}$, 
    where $\mathcal{E}\in\mathcal{M}$, $\mathcal{F}\subset N$
    for $\mu(N)=0$.
    By the definition of $\bar{\mu}$, 
    $\bar{\mu}(\mathcal{S})=\mu(\mathcal{E})\ge 0$. 
    So $\bar{\mu}:\bar{\mathcal{M}}\rightarrow [0,\infty]$.

    Second, we show the additive. 
    Choose $\{\mathcal{S}_{i}\}_{i=1}^{\infty}\subset\bar{\mathcal{M}}$ 
    with disjoint sets $\mathcal{S}_{i}$, i.e. 
    $\mathcal{S}_{i}=\mathcal{E}_{i}\cup\mathcal{F}_{i}$
    and $\E_{i}\cap\F_{i}=\emptyset$, $\{\E_{i}\}$ disjoint, 
    $\F_{i}\subset N_{i}$ with $\mu(N_{i})=0$. 
    Then: 
    \begin{displaymath}
        \bar{\mu}\left(\cup_{i=1}^{\infty}\Sc_{i}\right)
        =\bar{\mu}\left(\left(\cup_{i=1}^{\infty}\E_{i}\right)
        \cup\left(\cup_{i=1}^{\infty}\F_{i}\right)\right).
    \end{displaymath}
    As $\F_{i}\subset N_{i}$ with $\mu(N_{i})=0$, 
    $\cup_{i=1}^{\infty}\F_{i}\subset\cup_{i=1}^{\infty}N_{i}$
    with $\mu(\cup_{i=1}^{\infty}N_{i})=0$.
    So:
    \begin{displaymath}
        \bar{\mu}(\cup_{i=1}^{\infty}\Sc_{i})=\mu(\cp{i}\E_{i})
        =\sum_{i=1}^{\infty}\mu(\E_{i})=\sm{i}\bar{\mu}(\E_{i}\cup\F_{i})
        =\sm{i}\bar{\mu}(\Sc_{i}).
    \end{displaymath}

    Third, by the definition, $\bar{\mu}(\emptyset)=\mu(\emptyset)=0$. 
    So $\bar{\mu}$ is a measure. And for $N$ s.t. $\bar{\mu}(N)=0$, 
    i.e. $\exists \tilde{N}$ s.t. $N\subset\tilde{N}$, 
    $\mu(\tilde{N})=0$, we have: $\forall \F\subset N$, 
    $\F\subset\tilde{N}$, i.e. 
    $\bar{\mu}(\F)=\bar{\mu}(\emptyset\cup\F)=0$. 
    So $\bar{\mu}$ is complete.

    Then we show the uniqueness. Assume $\tilde{\mu}$ is a complete 
    measure on $\bar{\mathcal{M}}$, then:
    \begin{displaymath}
        \tilde{\mu}(\E)\le\tilde{\mu}(\E\cup\F)\le
        \tilde{\mu}(\E)+\tilde{\mu}(\F).
    \end{displaymath}
    As $\forall\F\subset N$ s.t. $\mu(N)=0$, $\tilde{\mu}(\F)=0$, 
    we can see $\tilde{\mu}(\E\cup\F)=\tilde{\mu}(\E)$, it means 
    $\tilde{\mu}=\bar{\mu}$.\qedhere
\end{proof}
\begin{homeworkProblem}
    A finitely additive measure $\mu$ is a measure iff it is 
    continuous from below. If $\mu(X)<\infty$, $\mu$ is a measure 
    iff it is continuous from above.
\end{homeworkProblem}
\begin{proof}
    If $\mu$ is a measure, Theorem 1.8 shows the continuity from 
    below, and the continuity from above in the case $\mu(X)<\infty$.
    So it suffices to show the opposite direction.

    If $\mu$ be finitely additive and continuous from below, 
    choose disjoint sets $\{\E_{i}\}_{i=1}^{\infty}$, 
    mark $\F_{i}:=\cup_{j=1}^{i}\E_{j}$, then 
    \begin{displaymath}
        \F_{1}\subset\F_{2}\subset\cdots\subset\F_{n}
        \subset\cdots.
    \end{displaymath}
    $\mu$ is continuous from below, i.e. 
    \begin{displaymath}
        \mu(\cp{i}\E_{i})=\mu(\cp{i}\F_{i})
        =\lim_{n\rightarrow\infty}\mu(\F_{n}).
    \end{displaymath}
    By the finitely additivity,
    \begin{displaymath}
        \lim_{n\rightarrow\infty}\mu(\F_{n})=\lim_{n\rightarrow\infty}
        \sum_{i=1}^{n}\mu(\E_{i})
        =\sm{i}\mu(\E_{i}).
    \end{displaymath}
    It means $\mu$ is additive.

    If $\mu$ be finitely additive and continuous from above, 
    $\mu(X)<\infty$, 
    choose disjoint sets $\{\E_{i}\}_{i=1}^{\infty}$, 
    $\F_{i}:=\cup_{j=1}^{i}\E_{i}$, then:
    \begin{displaymath}
        \F_{1}^{c}\supset\F_{2}^{c}\supset\cdots
        \supset\F_{n}^{c}\supset\cdots,
    \end{displaymath}
    and $\mu(\F_{1}^{c})\le\mu(X)<\infty$. Then:
    \begin{displaymath}
        \mu(\cap_{i=1}^{\infty}\E_{i}^{c})
        =\mu(\cap_{i=1}^{\infty}\F_{i}^{c})
        =\lim_{n\rightarrow\infty}\mu(\F_{n}^{c})
        =\mu(X)-\lim_{n\rightarrow\infty}\mu(\F_{n})=\mu(X)-\sm{n}\mu(\E_{n}).
    \end{displaymath}
    The second step is derived by the continuity from above, 
    and the final step from the finitely additivity.

    So:
    \begin{displaymath}
        \sm{n}\mu(\E_{n})=\mu(X)-\mu(\cap_{i=1}^{\infty}\F_{i}^{c})
        =\mu((\cap_{i=1}^{\infty}\F_{i}^{c})^{c})
        =\mu(\cp{i}\F_{i})=\mu(\cp{i}\E_{i}).
    \end{displaymath}
    It means $\mu$ is additive.
\end{proof}
\begin{homeworkProblem}
    Every $\sigma$-finite measure is semifinite.
\end{homeworkProblem}
\begin{proof}
    $\forall\E\subset\M$ and $\mu(\E)=\infty$, we should construct 
    $\F\subset\E$ s.t. $0<\mu(\F)<\infty$.

    As $\mu$ be $\sigma$-finite, 
    $\exists\{\E_i\}_{i=1}^{\infty}\subset\M$ s.t. 
    $\cp{i}\E_{i}=X$, $\mu(\E_{i})<\infty$, 
    then $\E=\cp{i}(\E_{i}\cap\E)$.

    Mark $\F_{i}:=\E_{i}\cap\E$, as $\mu(\E_{i})<\infty$, 
    $\forall i$, $\mu(\F_{i})<\infty$. 
    If $\forall i$, $\mu(\F_{i})=0$, as $\E=\cp{i}\F_{i}$, 
    $\mu(\E)=0$, contradict!

    So $\exists j\in\mathbb{N}$ s.t. $\mu(\F_{j})>0$, 
    $\F_{j}\subset\E$ and $0<\mu(\F_{j})<\infty$.
\end{proof}
\begin{homeworkProblem}
    If $\mu$ is a semifinite measure and $\mu(E)=\infty$, 
    for any $C>0$ there exists $F\subset E$ 
    with $C<\mu(F)<\infty$.
\end{homeworkProblem}
\begin{proof}
    Proof by contradiction. 
    Assume $\exists C_{0}>0$ s.t. $\forall F\subset E$ 
    with $\mu(F)<\infty$, $\mu(F)<C_{0}$.
    Mark 
    \begin{displaymath}
        \mathcal{F}:=\{F|F\subset E, \mu(F)<C_{0}\},
    \end{displaymath}
    $\F$ is an ordered set with relation $\subset$. 
    By Zorn's Lemma, 
    there exists a maximum element $\bar{F}$ in $\mathcal{F}$, 
    and $\mu(\bar{F})<C_{0}$.
    
    As $\mu(E)=\infty$ and $\mu(\bar{F})<C_{0}$, 
    $\mu(E\setminus\bar{F})=\infty$. 
    $\mu$ be semifinite means 
    $\exists \mathcal{S}\subset E\setminus\bar{F}$, 
    $\mu(\mathcal{S})<\infty$.
    So $\mu(\bar{F}\cup\mathcal{S})<C_{0}+\mu(\mathcal{S})<\infty$.

    On the other hand, by assumption, 
    $\mu(\bar{F}\cup\mathcal{S})<\infty$ means 
    $\mu(\bar{F}\cup\mathcal{S})<C_{0}$, 
    i.e. $\bar{F}\cup\mathcal{S}\in\F$. 
    However, $\bar{F}$ is the maximum element in $\F$, 
    this leads to a contradiction, Q.E.D.\qedhere
\end{proof}
\begin{homeworkProblem}
    If $\mu^{*}$ is an outer measure on $X$ and 
    $\{A_{j}\}_{1}^{\infty}$ is a sequence of 
    disjoint $\mu^{*}$-measurable sets, then 
    $\mu^{*}(E\cap(\cp{j}A_{j}))=\sm{j}\mu^{*}(E\cap A_{j})$ 
    for any $E\subset X$.
\end{homeworkProblem}
\begin{proof}
    By Caratheodory Theorem, $\mu^{*}$-measurable sets 
    form a $\sigma$-algebra. 
    Mark $B_{n}:=\cup_{j=1}^{n}A_{j}$, 
    we show:
    \begin{equation}
        \label{eq:OuterMeasure}
        \forall n\in\mathbb{N},
        \;\mu^{*}(E\cap B_{n})
        =\sum_{j=1}^{n}\mu^{*}(E\cap A_{j}).
    \end{equation}
    Proof by induction. For $n=1$, $B_{1}=A_{1}$, 
    so \eqref{eq:OuterMeasure} holds. 
    Assume \eqref{eq:OuterMeasure} holds for $n=k$, consider $n=k+1$. 

    By the definition of $\mu^{*}$-measurable, 
    \begin{equation}
        \label{eq:Caratheodory}
        \begin{aligned}
            \mu^{*}(E)&=\mu^{*}(E\cap B_{n+1})
            +\mu^{*}(E\cap B_{n+1}^{c})\\
            &=\mu^{*}(E\cap B_{n+1})
            +\mu^{*}(E\cap B_{n}^{c}\cap A_{n+1}^{c})\\
            &=\mu^{*}(E\cap B_{n+1})+\mu^{*}(E\cap B_{n}^{c})
            -\mu^{*}(E\cap B_{n}^{c}\cap A_{n+1})\\
            &=\mu^{*}(E\cap B_{n+1})+\mu^{*}(E\cap B_{n}^{c})
            -\mu^{*}(E\cap A_{n+1}),
        \end{aligned}
    \end{equation}
    the final step from the fact that $\{A_{i}\}$ disjoint. 
    By \eqref{eq:Caratheodory} and the assumption,
    \begin{equation}
        \label{eq:Caratheodory2}
        \begin{aligned}
            \mu^{*}(E\cap B_{n+1})
            &=\mu^{*}(E)-\mu^{*}(E\cap B_{n}^{c})
            +\mu^{*}(E\cap A_{n+1})\\
            &=\mu^{*}(E\cap B_{n})+\mu^{*}(E\cap A_{n+1})\\
            &=\sum_{j=1}^{n+1}\mu^{*}(E\cap A_{j}).
        \end{aligned}
    \end{equation}
    So, by induction, \eqref{eq:OuterMeasure} 
    holds for all $n\in\mathbb{N}$.
    Then we consider three different cases.
    \begin{itemize}
        \item $\exists j_{0}$ such that 
        $\mu^{*}(E\cap A_{j_{0}})=\infty$.
        \item $\sm{j}\mu^{*}(E\cap A_{j})$ diverges, 
        but $\forall j\in\mathbb{N},$ $\mu^{*}(E\cap A_{j})<\infty$.
        \item $\sm{j}\mu^{*}(E\cap A_{j})$ converges.
    \end{itemize}
    Mark $A:=\cp{j}A_{j}$. 
    For the first case, 
    $\text{LHS}=\mu^{*}(E\cap A)\ge\mu^{*}(E\cap A_{j_0})
    =\infty$, $\text{RHS}\ge\mu^{*}(E\cap A_{j_{0}})=\infty$, 
    i.e. the result is true.

    For the second case, it's clear that:
    \begin{displaymath}
        \forall n\in\mathbb{N}, 
        \mu^{*}(E\cap A)\ge\mu^{*}(E\cap(\cup_{j=1}^{n}A_{j}))
        =\sum_{j=1}^{n}\mu^{*}(E\cap A_{j}).
    \end{displaymath}
    Set $n\rightarrow\infty$, it means $\text{LHS}=\text{RHS}=\infty$.

    For the third case, it suffices to show 
    \begin{equation}
        \label{eq:communte}
        \mu^{*}(E\cap A)=\lim_{n\rightarrow\infty}
        \mu^{*}(E\cap(\cup_{j=1}^{n}A_{j})).
    \end{equation}
    $\forall n\in\mathbb{N}$, it's clear that:
    \begin{displaymath}
        0\le \mu^{*}(E\cap A)-\mu^{*}(E\cap(\cup_{j=1}^{n}A_{j}))
        \le\mu^{*}(E\cap(\cup_{j=n+1}^{\infty}A_{j}))
        \le\sum_{j=n+1}^{\infty}\mu^{*}(E\cap A_{j}).
    \end{displaymath}
    As $\sm{j}\mu^{*}(E\cap A_{j})$ converges, 
    by Cauchy's convergence theorem, 
    $\lim_{n\rightarrow\infty}\sum_{j=n+1}^{\infty}
    \mu^{*}(E\cap A_{j})=0$. 
    Set $n\rightarrow\infty$, \eqref{eq:communte} holds. 
    Then set $n\rightarrow\infty$ on \eqref{eq:OuterMeasure}, 
    we complete the proof.
\end{proof}
\begin{homeworkProblem}
    Let $\mathcal{A}\subset\mathcal{P}(X)$ be an algebra, 
    $\mathcal{A}_{\sigma}$ the collection of countable unions 
    of sets in $\mathcal{A}$, and $\mathcal{A}_{\sigma\delta}$ 
    the collection of countable intersections of sets in 
    $\mathcal{A}_{\sigma}$. Let $\mu_{0}$ be a 
    premeasure on $\mathcal{A}$ and $\mu^{*}$ 
    the induced outer measure.
    \begin{enumerate}
        \item For any $E\subset X$ and $\epsilon>0$ 
        there exists $A\in\mathcal{A}_{\sigma}$ with $E\subset A$ 
        and $\mu^{*}(A)\le\mu^{*}(E)+\epsilon$.
        \item If $\mu^{*}(E)<\infty$, then $E$ is $\mu^{*}$-measurable 
        iff there exists $B\in\mathcal{A}_{\sigma\delta}$ 
        with $E\subset B$ and $\mu^{*}(B\setminus E)=0$.
        \item If $\mu_0$ is $\sigma$-finite, 
        the restriction $\mu^{*}(E)<\infty$ in $(b)$ is superfluous.
    \end{enumerate}
\end{homeworkProblem}
\begin{proof}
    (1) If $\mu^{*}(E)=\infty$, 
    just set $A=\cup_{\mathcal{S}\in\mathcal{A}}\mathcal{S}$. 
    So we only need  to check the case $\mu^{*}(E)<\infty$.
    By the definition of outer measure derived by premeasure,
    \begin{displaymath}
        \mu^{*}(E)=\inf_{(\cp{i}A_{i})\supset E}\sum\mu_{0}(A_{i}),
    \end{displaymath} 
    where $\{A_{i}\}$ are disjoint. 
    It means that $\forall\epsilon>0$, 
    $\exists\{A_{i}\}\subset\mathcal{A}$, 
    s.t. $E\subset\cup_{i=1}^{\infty}A_{i}$ 
    and $\mu^{*}(E)+\epsilon\ge\sum_{i=1}^{\infty}\mu_{0}(A_{i})$.
    Set $A:=\cp{i}A_{i}$, we can see $A\in\mathcal{A}_{\sigma}$, 
    and $\mu^{*}(A)=\sum_{i=1}^{\infty}\mu_{0}(A_{i})$. 
    It completes the proof.\qed

    (2) Before the proof, we introduce 
    the following three lemmas first. 
    \begin{lemma}
        \label{lem:AsigmadeltaFormSA}
        $\mathcal{A}_{\sigma\delta}$ is a subset of the $\sigma$-algebra 
        generated by elements in $\mathcal{A}$.
    \end{lemma}
    \begin{proof}
        $\sigma$-algebra $\mathcal{M}(\mathcal{A})$ is closed 
        under countable unions and intersections, 
        and $\mathcal{A}\subset\mathcal{M}(\mathcal{A})$, 
        which means 
        $\mathcal{A}_{\sigma\delta}\subset\mathcal{M}(\mathcal{A})$. 
    \end{proof}
    \begin{lemma}
        \label{lem:AsdMS}
        $\forall\mathcal{S}\in\mathcal{A}_{\sigma\delta}$, 
        $\mathcal{S}$ is a $\mu^*$-measurable set.
    \end{lemma}
    \begin{proof}
        By Caratheodory's Theorem, the $\mu^*$-measurable sets 
        form a $\sigma$-algebra. 
        On the other hand, by Proposition 1.13, 
        $\forall$ $A\in\mathcal{A}$, $A$ is $\mu^*$-measurable.
        Choose $\mathcal{T}:=\{\mathcal{S}:\mathcal{S}\text{ is }
        \mu^*\text{-measurable}\}$, 
        by Lemma \ref{lem:AsigmadeltaFormSA}, 
        $\mathcal{A}_{\sigma\delta}\subset\mathcal{T}$. Q.E.D.
    \end{proof}
    \begin{lemma}
        \label{lem:ZeroMeasurable}
        $\mu^{*}(A)=0$ means $A$ is $\mu^{*}$-measurable.
    \end{lemma}
    \begin{proof}
        It suffices to show that $\forall\mathcal{S}\in\mathcal{P}(X)$, 
        $\mu^{*}(\mathcal{S})=\mu^{*}(S\cap A^{c})$. 
        $\forall$ disjoint sets $\{\mathcal{A}_{i}\}\subset\mathcal{A}$ 
        satisfies $\cup_{i=1}^{\infty}
        \mathcal{A}_{i}\supset\mathcal{S}$, 
        choose $\tilde{\mathcal{A}}_{i}:=\mathcal{A}_{i}\setminus A$, 
        it means that 
        $\cp{i}\tilde{\mathcal{A}}_{i}\supset\mathcal{S}\cap A^c$, 
        and $\sm{i}\mu_{0}(\mathcal{A}_{i})
        =\sm{i}\mu_{0}(\tilde{\mathcal{A}}_{i})$. 
        It completes the proof.
    \end{proof}
    Now we continue the proof.

    $"\Rightarrow"$: 
    As $\mu^{*}(E)<\infty$, 
    by $(1)$, we can choose a sequence of sets 
    $A_{n}\in\mathcal{A}_{\sigma}$, s.t. 
    \begin{itemize}
        \item $E\subset A_{n}$.
        \item $\mu^*(A_{n})\le\mu^{*}(E)+\frac{1}{n}$.
    \end{itemize}
    Choose $B:=\cap_{n=1}^{\infty}A_{n}$, 
    it's clear that $B\in\mathcal{A}_{\sigma\delta}$, $E\subset B$. 
    WLOG, we assume the sequence $\{A_{n}\}$ is decreasing. 
    Then:
    \begin{displaymath}
        0\le\mu^{*}(B\setminus E)=\mu^{*}
        (\cap_{n=1}^{\infty}A_{n}\setminus E)
        =\lim_{n\rightarrow\infty}\mu^{*}(A_{n}\setminus E)
        \le\lim_{n\rightarrow\infty}\frac{1}{n}
        =0.
    \end{displaymath}
    It completes the proof.

    $"\Leftarrow"$: Choose $F\in\mathcal{P}(X)$. 
    By the monotony of $\mu^{*}$, 
    \begin{displaymath}
        \mu^{*}(F\cap E)+\mu^{*}(F\cap E^{c})\le \mu^{*}(F\cap B)
        +\mu^{*}(F\cap E^c).
    \end{displaymath}
    By Lemma \ref{lem:AsdMS}, $B$ is $\mu^*$-measurable, i.e. 
    $\mu^{*}(F)=\mu^{*}(F\cap B)+\mu^{*}(F\cap B^{c})$. 
    So it suffices to show 
    $\mu^{*}(F\cap E^{c})=\mu^{*}(F\cap B^{c})$. 
    We have:
    \begin{displaymath}
        \begin{aligned}
        \mu^{*}(F\cap B^c)&=\mu^{*}(F\cap E^{c}\cap(B\setminus E)^{c})\\
        &=\mu^{*}(F\cap E^{c})+\mu^{*}(F\cap E^{c}\cap(B\setminus E))\\
        &=\mu^{*}(F\cap E^{c}).
        \end{aligned}
    \end{displaymath}
    The first step is from $B=E\cup B\setminus E$, 
    The second from the fact that $\mu^{*}(B\setminus E)=0$ 
    and Lemma \ref{lem:ZeroMeasurable}, 
    and the third from the monotony of $\mu^{*}$.

    This completes the proof.\qed
    
    (3) $\mu_{0}$ be $\sigma$-finite means 
    $\exists\{A_{i}\}\subset\mathcal{A}$ s.t. 
    \begin{itemize}
        \item $X=\cp{i}A_{i}$.
        \item $\mu_{0}(A_{i})<\infty$.
    \end{itemize}

    $"\Rightarrow"$: $E$ is $\mu^*$-measurable means 
    $A_{i}\cap E$ is $\mu^*$-measurable. 
    By $(2)$, $\exists B_{i}\in \mathcal{A}_{\sigma\delta}$ 
    with $X_{i}\cap E\subset B_{i}$ 
    and $\mu^{*}(B_{i}\setminus(X_{i}\cap E))=0$. 
    Mark $B:=\cp{i}B_{i}$, it's clear that $E\subset\cp{i}B_{i}=B$ and 
    \begin{displaymath}
        \mu^{*}(B\setminus E)\le
        \sm{i}\mu^{*}(B_{i}\setminus(X_{i}\cap E))
        =0.
    \end{displaymath}

    $"\Leftarrow"$: $E\subset B$ means 
    $A_{i}\cap E\subset A_{i}\cap B$, then 
    \begin{displaymath}
        \mu^{*}((A_{i}\cap B)\setminus(A_{i}\cap E))
        \le\mu^{*}(B\cap E)=0.
    \end{displaymath}
    On the other hand, for $B\in\mathcal{A}_{\sigma\delta}$, 
    $A_{i}\cap B\in \mathcal{A}_{\sigma\delta}$. 
    By $(2)$, $X_{i}\cap E$ is $\mu^{*}$-measurable.
    By Caratheodory's Theorem, 
    $\cp{i}(X_{i}\cap E)=E$ is $\mu^{*}$-measurable.
\end{proof}
\begin{homeworkProblem}
    Let $\mu^*$ be an outer measure on $X$ induced from a 
    finite premeasure $\mu_{0}$. 
    If $E\subset X$, define the \textbf{inner measure} of $E$ 
    to be $\mu_{*}(E)=\mu_{0}(X)-\mu^{*}(E^{c})$. 
    Then $E$ is $\mu^{*}$-measurable iff $\mu^{*}(E)=\mu_{*}(E)$. 
\end{homeworkProblem}
\begin{proof}
    $"\Rightarrow"$: If $E$ is $\mu^{*}$-measurable 
    and $\mu_0$ be finite, 
    by Problem 6$(3)$, $\exists B\in\mathcal{A}_{\sigma\delta}$ 
    s.t. $\mu^{*}(B\setminus E)=0$.

    By Lemma \ref{lem:AsdMS}, $B$ is $\mu^{*}$-measurable, i.e. 
    $\mu_{0}(X)=\mu^{*}(B)+\mu^{*}(B^{c})$. 
    On the other hand, we have:
    \begin{itemize}
        \item $\mu^{*}(E)\le\mu^{*}(B)=\mu^{*}(E\cup(B\setminus E))
        \le\mu^{*}(E)+\mu^{*}(B\setminus E)=\mu^{*}(E)$.
        \item $\mu^{*}(B^{c})=\mu^{*}(E^{c}\cap(B\setminus E)^{c})
        =\mu^{*}(E^{c})-\mu^{*}(E^{c}\cap(B\setminus E))
        =\mu^{*}(E^{c})$,
        this equality is derived from Lemma \ref{lem:ZeroMeasurable}.
    \end{itemize}
    So $\mu_{0}(X)=\mu^{*}(E)+\mu^{*}(E^{c})=\mu^{*}(E)+\mu_{*}(E)$.\qed

    $"\Leftarrow"$: Since $\mu^{*}$ be finite, $\mu^{*}(E)<\infty$. 
    Then by Problem 6(1), 
    $\exists$ $\{A_{n}\}\subset\mathcal{A}_{\sigma}$ s.t. 
    \begin{itemize}
        \item $E\subset A_{n}$.
        \item $\mu^{*}(A_{n})\le\mu^{*}(E)+\frac{1}{n}$.
    \end{itemize}
    Mark $B:=\cap_{n=1}^{\infty}A_{n}\in\mathcal{A}_{\sigma\delta}$, 
    by Lemma \ref{lem:AsdMS}, $B$ is $\mu^{*}$-measurable 
    and $E^{c}\supset B^{c}$, so:
    \begin{displaymath}
        \mu^{*}(E^{c})=\mu^{*}(E^{c}\cap B)+\mu^{*}(B^{c}).
    \end{displaymath}
    It means:
    \begin{displaymath}
        \begin{aligned}
            \mu^{*}(B\setminus E)&=\mu_{0}(X)-\mu^{*}(E)-\mu^{*}(B^{c})\\
            &\le\frac{1}{n}+\mu_{0}(X)-\mu^{*}(A_{n})-\mu^{*}(B^{c})\\
            &\le\frac{1}{n}+\mu_{0}(X)-\mu^{*}(A_{n}\cup B^{c})\\
            &=\frac{1}{n}.
        \end{aligned}
    \end{displaymath}
    Choose $n\rightarrow\infty$, $\mu^{*}(B\setminus E)=0$. 
    By Problem 6(3), $E$ is $\mu^{*}$-measurable.
\end{proof}
\end{document}