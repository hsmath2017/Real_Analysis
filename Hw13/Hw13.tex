\documentclass{article}

\usepackage{fancyhdr}
\usepackage{extramarks}
\usepackage{amsmath}
\usepackage{amsthm}
\newtheorem{lemma}{Lemma}
\usepackage{amsfonts}
\usepackage{tikz}
\usepackage[plain]{algorithm}
\usepackage{algpseudocode}

\usetikzlibrary{automata,positioning}

%
% Basic Document Settings
%

\topmargin=-0.45in
\evensidemargin=0in
\oddsidemargin=0in
\textwidth=6.5in
\textheight=9.0in
\headsep=0.25in

\linespread{1.1}

\pagestyle{fancy}
\lhead{\hmwkAuthorName}
\chead{\hmwkClass\ (\hmwkClassInstructor\ \hmwkClassTime): \hmwkTitle}
\rhead{\firstxmark}
\lfoot{\lastxmark}
\cfoot{\thepage}

\renewcommand\headrulewidth{0.4pt}
\renewcommand\footrulewidth{0.4pt}

\setlength\parindent{0pt}

%
% Create Problem Sections
%

\newcommand{\enterProblemHeader}[1]{
    \nobreak\extramarks{}{Problem \arabic{#1} continued on next page\ldots}\nobreak{}
    \nobreak\extramarks{Problem \arabic{#1} (continued)}{Problem \arabic{#1} continued on next page\ldots}\nobreak{}
}

\newcommand{\exitProblemHeader}[1]{
    \nobreak\extramarks{Problem \arabic{#1} (continued)}{Problem \arabic{#1} continued on next page\ldots}\nobreak{}
    \stepcounter{#1}
    \nobreak\extramarks{Problem \arabic{#1}}{}\nobreak{}
}

\setcounter{secnumdepth}{0}
\newcounter{partCounter}
\newcounter{homeworkProblemCounter}
\setcounter{homeworkProblemCounter}{1}
\nobreak\extramarks{Problem \arabic{homeworkProblemCounter}}{}\nobreak{}

%
% Homework Problem Environment
%
% This environment takes an optional argument. When given, it will adjust the
% problem counter. This is useful for when the problems given for your
% assignment aren't sequential. See the last 3 problems of this template for an
% example.
%
\newenvironment{homeworkProblem}[1][-1]{
    \ifnum#1>0
        \setcounter{homeworkProblemCounter}{#1}
    \fi
    \section{Problem \arabic{homeworkProblemCounter}}
    \setcounter{partCounter}{1}
    \enterProblemHeader{homeworkProblemCounter}
}{
    \exitProblemHeader{homeworkProblemCounter}
}

%
% Homework Details
%   - Title
%   - Due date
%   - Class
%   - Section/Time
%   - Instructor
%   - Author
%

\newcommand{\hmwkTitle}{Homework\ \#13}
\newcommand{\hmwkDueDate}{Jun 4, 2024}
\newcommand{\hmwkClass}{Real Analysis}
\newcommand{\hmwkClassTime}{Tuesday}
\newcommand{\hmwkClassInstructor}{Professor Yakun Xi}
\newcommand{\hmwkAuthorName}{\textbf{Shuang Hu}}

%
% Title Page
%

\title{
    \vspace{2in}
    \textmd{\textbf{\hmwkClass:\ \hmwkTitle}}\\
    \normalsize\vspace{0.1in}\small{Due\ on\ \hmwkDueDate\ at 10:00am}\\
    \vspace{0.1in}\large{\textit{\hmwkClassInstructor\ \hmwkClassTime}}
    \vspace{3in}
}

\author{\hmwkAuthorName}
\date{}

\renewcommand{\part}[1]{\textbf{\large Part \Alph{partCounter}}\stepcounter{partCounter}\\}

%
% Various Helper Commands
%

% Useful for algorithms
\newcommand{\alg}[1]{\textsc{\bfseries \footnotesize #1}}

% For derivatives
\newcommand{\deriv}[1]{\frac{\mathrm{d}}{\mathrm{d}x} (#1)}

% For partial derivatives
\newcommand{\pderiv}[2]{\frac{\partial}{\partial #1} (#2)}

% Integral dx
\newcommand{\dx}{\mathrm{d}x}

% Alias for the Solution section header
\newcommand{\solution}{\textbf{\large Solution}}
\newcommand{\norm}[1]{\|#1\|}
% Probability commands: Expectation, Variance, Covariance, Bias
\newcommand{\Var}{\mathrm{Var}}
\newcommand{\Cov}{\mathrm{Cov}}
\newcommand{\Bias}{\mathrm{Bias}}
\newcommand{\supp}{\text{supp}}
\newcommand{\Rn}{\mathbb{R}^{n}}
\newcommand{\dif}{\mathrm{d}}
\newcommand{\avg}[1]{\left\langle #1 \right\rangle}
\newcommand{\difFrac}[2]{\frac{\dif #1}{\dif #2}}
\newcommand{\pdfFrac}[2]{\frac{\partial #1}{\partial #2}}
\newcommand{\OFL}{\mathrm{OFL}}
\newcommand{\UFL}{\mathrm{UFL}}
\newcommand{\fl}{\mathrm{fl}}
\newcommand{\Eabs}{E_{\mathrm{abs}}}
\newcommand{\Erel}{E_{\mathrm{rel}}}
\newcommand{\DR}{\mathcal{D}_{\widetilde{LN}}^{n}}
\newcommand{\add}[2]{\sum_{#1=1}^{#2}}
\newcommand{\innerprod}[2]{\left<#1,#2\right>}
\newcommand{\Sc}{\mathcal{S}}
\newcommand{\F}{\mathcal{F}}
\newcommand{\E}{\mathcal{E}}
\newcommand{\A}{\mathcal{A}}
\newcommand{\cp}[2]{\cup_{#1=1}^{#2}}
\newcommand{\sm}[2]{\sum_{#1=1}^{#2}}
\newcommand{\M}{\mathcal{M}}
\newcommand{\Lc}{\mathcal{L}}
\newcommand\tbbint{{-\mkern -16mu\int}}
\newcommand\tbint{{\mathchar '26\mkern -14mu\int}}
\newcommand\dbbint{{-\mkern -19mu\int}}
\newcommand\dbint{{\mathchar '26\mkern -18mu\int}}
\newcommand\bint{
{\mathchoice{\dbint}{\tbint}{\tbint}{\tbint}}
}
\newcommand\bbint{
{\mathchoice{\dbbint}{\tbbint}{\tbbint}{\tbbint}}
}
\begin{document}
\maketitle
\pagebreak
\begin{homeworkProblem}
    If $g\in L^{\infty}$, the operator $T$ defined by $Tf=fg$ 
    is bounded on $L^{p}$ for $1\le p\le \infty$. Its operator norm 
    is at most $\norm{g}_{\infty}$, with equality if $\mu$ 
    is semifinite.
\end{homeworkProblem}
\begin{proof}
    Since 
    \begin{displaymath}
        \begin{array}{rl}
        \norm{Tf}_{p}&=\left(\int_{X}|fg|^{p}\right)^{\frac{1}{p}}\\
        &\le\left(\norm{g}_{\infty}^{p}\int_{X}|f|^{p}\right)
        ^\frac{1}{p}\\
        &=\norm{g}_{\infty}\norm{f}_{p},
        \end{array}
    \end{displaymath}
    it's clear that $\norm{T}\le\norm{g}_{\infty}$. 
    
    If $\mu$ is 
    semifinite, i.e. $\exists E\subset X$ such that $\mu(E)<\infty$, 
    then $\forall\epsilon>0$, $\exists$ set $E_{\epsilon}$ s.t. 
    \begin{displaymath}
        \left\{
            \begin{array}{rl}
                0<\mu(E_{\epsilon})<\infty,\\
                \forall x\in E_{\epsilon},\quad 
                \norm{g}_{\infty}-\epsilon<|g(x)|<\norm{g}_{\infty}.
            \end{array}
        \right.
    \end{displaymath}
    We choose $f:=\chi_{E_{\epsilon}}$, derive 
    $\norm{T}\ge\norm{g}_{\infty}-\epsilon$, 
    so $\norm{T}=\norm{g}_{\infty}$.
\end{proof}
\begin{homeworkProblem}
    Suppose that $\mu,\nu$ are positive finite measures on $(X,\M)$, 
    and let $\lambda=\mu+\nu$.
    \begin{enumerate}
        \item The map $f\mapsto \int f\dif\nu$ is a bounded linear 
        functional on $L^{2}(\lambda)$, so $\int f\dif\nu
        =\int fg\dif\lambda$ for some $g\in L^{2}(\lambda)$. 
        Equivalently, $\int f(1-g)\dif\nu=\int fg\dif\mu$ for 
        $f\in L^{2}(\lambda)$.
        \item $0\le g\le 1$ $\lambda$-a.e., so we may assume 
        $0\le g\le 1$ everywhere. 
        \item Let $A=\{x:g(x)<1\}$, $B=\{x:g(x)=1\}$, and set 
        $\nu_{a}(E):=\nu(A\cap E)$, $\nu_{s}(E):=\nu(B\cap E)$. 
        Then $\nu_{s}\perp\mu$ and $\nu_{a}\ll\mu$; in fact, 
        $\dif\nu_{a}=g(1-g)^{-1}\chi_{A}\dif\mu$.
    \end{enumerate}
\end{homeworkProblem}
\begin{proof}
    $(1)$ Linear: just by the definition. 

    Bounded: since $\nu(X)<\infty$, by Cauchy-Schwarz inequality:
    \begin{displaymath}
        |Tf|=\left|\int_{X}f\dif\nu\right|
        \le\left(\int_{X}|f|^{2}\dif\nu\int_{X}1\dif\nu\right)
        ^{\frac{1}{2}}
        =\sqrt{\nu(X)}\norm{f}_{L^2}.
    \end{displaymath}
    So $T$ is a bounded linear operator on $L^{2}(\lambda)$. 
    By Riesz's Representation Theorem, there exists 
    $g\in L^{2}(\lambda)$ such that 
    \begin{equation}
        \label{equ:RepTf}
        Tf=\int f\dif\nu=\int fg\dif\lambda.
    \end{equation}
    Equivalently, 
    \begin{equation}
        \label{equ:Repdifmu}
        \int f(1-g)\dif\nu=\int f\dif\nu
        -\int fg\dif\nu
        =\int fg(\dif\lambda-\dif\nu)
        =\int fg\dif\mu.
    \end{equation}

    $(2)$ Assume $g<0$ on $E_1$ with $\lambda(E_1)>0$, 
    by \eqref{equ:RepTf}, 
    \begin{displaymath}
        \int\chi_{E_1}\dif\nu=\int \chi_{E_1}g\dif\lambda.
    \end{displaymath}
    If $\nu(E_1)>0$, it means $\int_{E_1}g\dif\lambda>0$, 
    but $g<0$ on $E_1$ with $\lambda(E_1)>0$, contradict! 

    If $\nu(E_1)=0$, i.e. $\mu(E_1)=\lambda(E_1)>0$, 
    then 
    \begin{displaymath}
        \int_{E_1}\dif\nu=\int g\dif\lambda=0.
    \end{displaymath}
    Contradict to $g<0$ and $\lambda(E_1)>0$. So $g\ge 0 \lambda$ a.e..

    Assume $g>1$ on $E_2$ with $\lambda(E_2)>0$, then 
    \begin{displaymath}
        \int_{E_2}\dif\nu=\int\chi_{E_2}g\dif\lambda
        \Rightarrow \nu(E_2)=\int_{E_2}g\dif\lambda.
    \end{displaymath}
    What's more, $\lambda=\mu+\nu$ and $\mu,\nu$ positive, 
    it means 
    \begin{displaymath}
        \int E_2 g\dif\lambda>\int E_2\dif\lambda=\lambda(E_2)
        \ge\nu(E_2),
    \end{displaymath}
    contradict! So $\lambda(E_2)=0$. i.e. $0\le g\le 1$ $\lambda$-a.e..

    $(3)$ On $B$, if $E\subset B$, 
    \begin{displaymath}
        \nu_{s}(E)=\nu(B\cap E)=\nu(E)
        =\int_{E}\dif\nu
        =\int\chi_{E}\dif\nu
        =\int\chi_{E}g\dif\lambda
        =\int\chi_{E}\dif\lambda.
    \end{displaymath}
    Since $\lambda=\mu+\nu$, it means $\mu(E)=0$. 
    So $\mu=0$ on $B$. 

    If $E\subset A$, $\nu_{s}(E)=\nu(B\cap E)=\nu(\emptyset)=0$. 
    Since $A\cup B=X$, $A\cap B=\emptyset$, $\nu_{s}\perp\mu$. 

    If $\mu(E)=0$, 
    \begin{displaymath}
        \begin{array}{rl}
            \lambda(E)=\nu(E)&\Rightarrow 
            \int \chi_{E}\dif\lambda=\int\chi_{E}\dif\nu
            =\int\chi_{E}g\dif\lambda\\
            &\Rightarrow g=1\text{ a.e. on }E\\
            &\Rightarrow \nu_{a}(E)=0\Rightarrow\nu_{a}\ll\mu.
        \end{array}
    \end{displaymath}

    Finally:
    \begin{displaymath}
        \begin{array}{rl}
            \int_{E}\dif\nu_a&=
            \nu_{a}(E)=\nu(A\cap E)
            =\int_{A\cap E}\dif\nu
            =\int_{E}\chi_{A}\dif\nu
            =\int_{E}\chi_{A}g\dif\lambda\\
            &=\int_{E}\chi_{A}g(\dif\mu+\dif\nu)
            =\int_{E}\chi_{A}g\dif\nu+\int_{E}\chi_{A}g\dif\mu
            =\int_{E}\chi_{A}\frac{g}{1-g}\dif\mu.
        \end{array}
    \end{displaymath}
    So $\dif\nu_{a}=g(1-g)^{-1}\chi_{A}\dif\mu$.
\end{proof}
\begin{homeworkProblem}
    TBD
\end{homeworkProblem}
\begin{homeworkProblem}
    Complete the proof of Theorem 6.18 for the cases $p=1$ 
    and $p=\infty$.
\end{homeworkProblem}
\begin{proof}
    For $p=\infty$:
    \begin{displaymath}
        \begin{array}{rl}
            |Tf(x)|&=\left|\int K(x,y)f(y)\dif\nu(y)\right|\\
            &\le\int|K(x,y)||f(y)|\dif\nu(y)\\
            &\le\norm{f}_{\infty}\int|K(x,y)|\dif\nu(y)
            \le C\norm{f}_{\infty}.
        \end{array}
    \end{displaymath}
    For $p=1$: 
    \begin{displaymath}
        \begin{array}{rl}
            \int |Tf(x)|\dif\mu(x)&=\int|\int K(x,y)f(y)\dif\nu(y)|
            \dif\mu(x)\\
            &\le\int\int|K(x,y)||f(y)|\dif\nu(y)\dif\mu(x)\\
            &\le\int\left(\int |K(x,y)|\dif\mu(x)\right)|f(y)|\dif\nu(y)\\
            &\le C\int|f(y)|\dif\nu(y)=C\norm{f}_1.
        \end{array}
    \end{displaymath}
    So $Tf\in L^1$, i.e. $Tf$ converges absolutely.
\end{proof}
\begin{homeworkProblem}
    Suppose that $1\le p_j\le\infty$ and $\sm{j}{n}p_{j}^{-1}=r^{-1}\le 1$. 
    If $f_{j}\in L^{p_j}$ for $j=1,\ldots,n$, then 
    $\prod_{1}^{n}f_j\in L^r$ and 
    $\norm{\prod_{1}^{n}f_j}_{r}\le \prod_{1}^{n}\norm{f_j}_{p_j}$. 
\end{homeworkProblem}
\begin{proof}
    First, consider the case for $n=2$, 
    since $1=\frac{r}{p_1}+\frac{r}{p_2}$, by Holder's inequality:
    \begin{displaymath}
        \norm{fg}_{r}^{r}=\int|fg|^r
        =\norm{f^{r}g^{r}}_{1}
        \le\norm{f^{r}}_{\frac{p_1}{r}}\norm{g^r}_{\frac{p_2}{r}}
        =\left(\int|f|^{p_1}\right)^{\frac{r}{p_1}}
        \left(\int|g|^{p_2}\right)^{\frac{r}{p_2}}.
    \end{displaymath}
    So $\norm{fg}_r\le\norm{f}_{p_1}\norm{g}_{p_2}$.

    Then using induction on $n$, assume the result holds for $n=k$. 
    When $n=k+1$, it means $\sm{i}{k+1}\frac{1}{p_i}=\frac{1}{r}$, 
    \begin{displaymath}
        \norm{\prod_{i=1}^{k+1}f_i}_{r}
        =\norm{f_{k+1}(\prod_{i=1}^{k}f_{i})}_{r}
        \le\norm{f_{k+1}}_{p_{k+1}}\norm{\prod_{i=1}^{n}f_{i}}
        _{\frac{1}{\frac{1}{r}-\frac{1}{p_{n+1}}}}.
    \end{displaymath}
    Since $\sm{i}{n}\frac{1}{p_i}=\frac{1}{r}-\frac{1}{p_{n+1}}$, 
    \begin{displaymath}
        \norm{\prod_{i=1}^{n}f_{i}}
        _{\frac{1}{\frac{1}{r}-\frac{1}{p_{n+1}}}}
        \le\prod_{i=1}^{n}\norm{f_{i}}_{p_{i}}.
    \end{displaymath}
    It completes the proof.
\end{proof}
\begin{homeworkProblem}
    Suppose that $(X,\M,\mu)$ and $(Y,\mathcal{N},\nu)$ are 
    $\sigma$-finite measure spaces and $K\in L^{2}(\mu\times\nu)$. 
    If $f\in L^{2}(\nu)$, the integral $Tf(x)=\int K(x,y)f(y)\dif\nu(y)$ 
    converges absolutely for a.e. $x\in X$; moreover, $Tf\in L^2(\mu)$ 
    and $\norm{Tf}_{2}\le\norm{K}_{2}\norm{f}_{2}$.
\end{homeworkProblem}
\begin{proof}
    Since $K\in L^{2}(\mu\times\nu)$, 
    \begin{displaymath}
        |Tf(x)|\le\int|K(x,y)f(y)|\dif\nu(y)
        \le\left(\int (K(x,y))^{2}\dif\nu(y)\right)^{\frac{1}{2}}
        \left(\int|f(y)|^{2}\dif\nu(y)\right)^{\frac{1}{2}}
        <\infty
    \end{displaymath}
    for a.e. $x$. Moreover:
    \begin{displaymath}
        \begin{array}{rl}
            \norm{Tf}_{2}^{2}&=\int |Tf(x)|^{2}\dif\mu(x)\\
            &=\int\left(\int K(x,y)f(y)\dif\nu(y)\right)^2\dif\mu(x)\\
            &\le\int\left(\int|K(x,y)|^{2}\dif\nu(y)
            \int|f(y)|^{2}\dif\nu(y)\right)\dif\mu(x)\\
            &=\norm{f}_{2}^{2}\iint |K(x,y)|^{2}\dif\nu(y)\dif\mu(x)\\
            &=\norm{f}_{2}^{2}\norm{K}_{2}^{2}.
        \end{array}
    \end{displaymath}
    So $\norm{Tf}_{2}\le\norm{K}_{2}\norm{f}_{2}$.
\end{proof}
\end{document}