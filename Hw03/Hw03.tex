\documentclass{article}

\usepackage{fancyhdr}
\usepackage{extramarks}
\usepackage{amsmath}
\usepackage{amsthm}
\newtheorem{lemma}{Lemma}
\usepackage{amsfonts}
\usepackage{tikz}
\usepackage[plain]{algorithm}
\usepackage{algpseudocode}

\usetikzlibrary{automata,positioning}

%
% Basic Document Settings
%

\topmargin=-0.45in
\evensidemargin=0in
\oddsidemargin=0in
\textwidth=6.5in
\textheight=9.0in
\headsep=0.25in

\linespread{1.1}

\pagestyle{fancy}
\lhead{\hmwkAuthorName}
\chead{\hmwkClass\ (\hmwkClassInstructor\ \hmwkClassTime): \hmwkTitle}
\rhead{\firstxmark}
\lfoot{\lastxmark}
\cfoot{\thepage}

\renewcommand\headrulewidth{0.4pt}
\renewcommand\footrulewidth{0.4pt}

\setlength\parindent{0pt}

%
% Create Problem Sections
%

\newcommand{\enterProblemHeader}[1]{
    \nobreak\extramarks{}{Problem \arabic{#1} continued on next page\ldots}\nobreak{}
    \nobreak\extramarks{Problem \arabic{#1} (continued)}{Problem \arabic{#1} continued on next page\ldots}\nobreak{}
}

\newcommand{\exitProblemHeader}[1]{
    \nobreak\extramarks{Problem \arabic{#1} (continued)}{Problem \arabic{#1} continued on next page\ldots}\nobreak{}
    \stepcounter{#1}
    \nobreak\extramarks{Problem \arabic{#1}}{}\nobreak{}
}

\setcounter{secnumdepth}{0}
\newcounter{partCounter}
\newcounter{homeworkProblemCounter}
\setcounter{homeworkProblemCounter}{1}
\nobreak\extramarks{Problem \arabic{homeworkProblemCounter}}{}\nobreak{}

%
% Homework Problem Environment
%
% This environment takes an optional argument. When given, it will adjust the
% problem counter. This is useful for when the problems given for your
% assignment aren't sequential. See the last 3 problems of this template for an
% example.
%
\newenvironment{homeworkProblem}[1][-1]{
    \ifnum#1>0
        \setcounter{homeworkProblemCounter}{#1}
    \fi
    \section{Problem \arabic{homeworkProblemCounter}}
    \setcounter{partCounter}{1}
    \enterProblemHeader{homeworkProblemCounter}
}{
    \exitProblemHeader{homeworkProblemCounter}
}

%
% Homework Details
%   - Title
%   - Due date
%   - Class
%   - Section/Time
%   - Instructor
%   - Author
%

\newcommand{\hmwkTitle}{Homework\ \#3}
\newcommand{\hmwkDueDate}{Mar 19, 2024}
\newcommand{\hmwkClass}{Real Analysis}
\newcommand{\hmwkClassTime}{Tuesday}
\newcommand{\hmwkClassInstructor}{Professor Yakun Xi}
\newcommand{\hmwkAuthorName}{\textbf{Shuang Hu}}

%
% Title Page
%

\title{
    \vspace{2in}
    \textmd{\textbf{\hmwkClass:\ \hmwkTitle}}\\
    \normalsize\vspace{0.1in}\small{Due\ on\ \hmwkDueDate\ at 10:00am}\\
    \vspace{0.1in}\large{\textit{\hmwkClassInstructor\ \hmwkClassTime}}
    \vspace{3in}
}

\author{\hmwkAuthorName}
\date{}

\renewcommand{\part}[1]{\textbf{\large Part \Alph{partCounter}}\stepcounter{partCounter}\\}

%
% Various Helper Commands
%

% Useful for algorithms
\newcommand{\alg}[1]{\textsc{\bfseries \footnotesize #1}}

% For derivatives
\newcommand{\deriv}[1]{\frac{\mathrm{d}}{\mathrm{d}x} (#1)}

% For partial derivatives
\newcommand{\pderiv}[2]{\frac{\partial}{\partial #1} (#2)}

% Integral dx
\newcommand{\dx}{\mathrm{d}x}

% Alias for the Solution section header
\newcommand{\solution}{\textbf{\large Solution}}
\newcommand{\norm}[1]{\|#1\|}
% Probability commands: Expectation, Variance, Covariance, Bias
\newcommand{\Var}{\mathrm{Var}}
\newcommand{\Cov}{\mathrm{Cov}}
\newcommand{\Bias}{\mathrm{Bias}}
\newcommand{\supp}{\text{supp}}
\newcommand{\Rn}{\mathbb{R}^{n}}
\newcommand{\dif}{\mathrm{d}}
\newcommand{\avg}[1]{\left\langle #1 \right\rangle}
\newcommand{\difFrac}[2]{\frac{\dif #1}{\dif #2}}
\newcommand{\pdfFrac}[2]{\frac{\partial #1}{\partial #2}}
\newcommand{\OFL}{\mathrm{OFL}}
\newcommand{\UFL}{\mathrm{UFL}}
\newcommand{\fl}{\mathrm{fl}}
\newcommand{\Eabs}{E_{\mathrm{abs}}}
\newcommand{\Erel}{E_{\mathrm{rel}}}
\newcommand{\DR}{\mathcal{D}_{\widetilde{LN}}^{n}}
\newcommand{\add}[2]{\sum_{#1=1}^{#2}}
\newcommand{\innerprod}[2]{\left<#1,#2\right>}
\newcommand{\Sc}{\mathcal{S}}
\newcommand{\F}{\mathcal{F}}
\newcommand{\E}{\mathcal{E}}
\newcommand{\A}{\mathcal{A}}
\newcommand{\cp}[2]{\cup_{#1=1}^{#2}}
\newcommand{\sm}[2]{\sum_{#1=1}^{#2}}
\newcommand{\M}{\mathcal{M}}
\newcommand\tbbint{{-\mkern -16mu\int}}
\newcommand\tbint{{\mathchar '26\mkern -14mu\int}}
\newcommand\dbbint{{-\mkern -19mu\int}}
\newcommand\dbint{{\mathchar '26\mkern -18mu\int}}
\newcommand\bint{
{\mathchoice{\dbint}{\tbint}{\tbint}{\tbint}}
}
\newcommand\bbint{
{\mathchoice{\dbbint}{\tbbint}{\tbbint}{\tbbint}}
}
\begin{document}
\maketitle
\pagebreak
\begin{homeworkProblem}
    Let $\A$ be the collection of finite unions of sets of the form 
    $(a,b]\cap\mathbb{Q}$ where $-\infty\le a<b\le\infty$.
    \begin{enumerate}
        \item $\A$ is an algebra on $\mathbb{Q}$.
        \item The $\sigma$-algebra generated by $\A$ is 
        $\mathcal{P}(\mathbb{Q})$.
        \item Define $\mu_{0}$ on $\A$ by $\mu_{0}(\emptyset)=0$ 
        and $\mu_{0}(A)=\infty$ for $A\neq\emptyset$. Then $\mu_{0}$ 
        is a premeasure on $\A$, and there is more than one measure 
        on $\mathcal{P}(\mathbb{Q})$ whose restriction to $\A$ 
        is $\mu_{0}$.
    \end{enumerate}
\end{homeworkProblem}
\begin{proof}
    $(1)$
    First, we show $\E:=\{(a,b]\cap\mathbb{Q}:
    -\infty\le a<b\le\infty\}$ forms an elementary family.
    If $a>b$, the notation $(a,b]:=\emptyset$.

    Set $\Sc_{1}:=(a,b]\cap\mathbb{Q}$, 
    $\Sc_2:=(x,y]\cap\mathbb{Q}$, 
    then $\Sc_1\cap\Sc_2=(\max(a,x),\min(b,y)]\cap\mathbb{Q}\in\E$.
    On the other hand, $\Sc_1^{c}=((-\infty,a]\cap\mathbb{Q})
    \cup((b,\infty]\cap\mathbb{Q})$, 
    which is a finite union of elements in $\E$. 
    So $\E$ is an elementary family. 

    Then, by Proposition 1.7, $\A$ is an algebra on $\mathbb{Q}$.\qed

    $(2)$ As $\A\subset\mathcal{P}(\mathbb{Q})$ and 
    $\mathcal{P}(\mathbb{Q})$ be a $\sigma$-algebra, 
    $\M(\A)\subset\mathcal{P}(\mathbb{Q})$. 
    It suffices to show $\mathcal{P}(\mathbb{Q})\subset\M(\A)$.
    
    Choose $\Sc\in\mathcal{P}(\mathbb{Q})$, write 
    $\Sc=\{q_{1},\ldots,q_{n},\ldots\}$. As $\forall q\in\mathbb{Q}$, 
    $(q-\frac{1}{n},q]\cap\mathbb{Q}\in\A$, i.e. 
    $\cap_{n=1}^{\infty}(q-\frac{1}{n},q]\cap\mathbb{Q}=\{q\}\in\M(\A)$, we can see:
    \begin{displaymath}
        \Sc=\cp{i}{\infty}\{q_{i}\}\in\M(\A).
    \end{displaymath}
    So $\mathcal{P}(\mathbb{Q})\subset\M(\A)$. \qed

    $(3)$ By definition, $\mu_{0}(\emptyset)=0$. 
    If $\emptyset\neq A=\cp{i}{\infty}A_i\in\A$ 
    with disjoint sets $A_i$, which implies 
    $\exists j\in\mathbb{N}$ s.t. $A_{j}\neq \emptyset$, then:
    \begin{displaymath}
        \mu_{0}(A)=\infty,\quad\sm{i}{\infty}\mu_{0}(A_{i})\ge\mu_{0}(A_{j})
        =\infty.
    \end{displaymath}
    So $\mu_{0}(\cp{i}{\infty}A_{i})=\sm{i}{\infty}\mu_{0}(A_i)$. 
    It means $\mu_{0}$ is a premeasure on $\A$. 

    Set $\mu_1,\mu_2:
    \mathcal{P}(\mathbb{Q})\rightarrow[0,\infty]$ satisfies: 
    \begin{itemize}
        \item $\mu_1(\emptyset)=\mu_2(\emptyset)=0$.
        \item For $\text{card}(\Sc)=\infty$, 
        $\mu_{1}(\Sc)=\mu_2(\Sc)=\infty$.
        \item For $\text{card}(\Sc)<\infty$, 
        $\mu_{1}(\Sc)=\infty$, $\mu_{2}(\Sc)=\text{card}(\Sc)$.
    \end{itemize}
    It's easy to check that $\mu_{1},\mu_{2}$ are distinct 
    measures on $\mathcal{P}(\mathbb{Q})$. 
    
    On the other hand, $\mathbb{Q}$ is dense in $\mathbb{R}$, 
    which means $\forall a<b$, 
    $\text{card}((a,b]\cap\mathbb{Q})=\infty$.
    So $\forall\Sc\in\A$ with $\Sc\neq\emptyset$, 
    $\mu_{1}(\Sc)=\mu_2(\Sc)=\infty$. 
    i.e. $\mu_1|_{\A}=\mu_{2}|_{\A}=\mu_{0}$.\qedhere
\end{proof}
\begin{homeworkProblem}
    Let $\mu$ be a finite measure on $(X,\M)$, 
    and let $\mu^{*}$ be the outer measure induced by $\mu$. 
    Suppose that $E\subset X$ satisfies $\mu^{*}(E)=\mu^{*}(X)$. 
    \begin{enumerate}
        \item If $A,B\in\M$ and $A\cap E=B\cap E$, 
        then $\mu(A)=\mu(B)$.
        \item Let $\M_{E}=\{A\cap E:A\in\M\}$, 
        and define the function $\nu$ on $\M_{E}$ defined by 
        $\nu(A\cap E)=\mu(A)$. Then $\M_{E}$ is a $\sigma$-algebra 
        on $E$ and $\nu$ is a measure on $\M_{E}$.
    \end{enumerate}
\end{homeworkProblem}
\begin{proof}
    $(1)$ First, we show $\mu(A\cap B^{c})=0$. 
    Assume $\mu(A\cap B^{c})>0$, 
    mark $\Sc:=A\cap B^{c}$. 
    If $\Sc\cap E=\emptyset$, 
    i.e. $E\subset\Sc^{c}$, 
    and $\mu(\Sc^{c})=\mu(X)-\mu(\Sc)<\mu(X)$, 
    it means $\mu^{*}(E)<\mu^{*}(X)$, contradict! 
    So $\Sc\cap E\neq \emptyset$. 

    On the other hand, as $A\cap E=B\cap E$, 
    \begin{displaymath}
        \emptyset=(A\cap E)\cap (B\cap E)^{c}
        =(A\cap E)\cap(B^{c}\cup E^{c})
        \supset A\cap B^{c}\cap E
        =\Sc\cap E,
    \end{displaymath}
    contradict! So $\mu(\Sc)=0$, 
    which means that $\mu(A)=\mu(A\cap B)$.

    In the same way, $\mu(A^{c}\cap B)=0$, i.e. $\mu(B)=\mu(A\cap B)$. 
    So $\mu(A)=\mu(B)$. \qed

    $(2)$ First, we show $(E,\M_{E})$ be a $\sigma$-algebra.

    Choose disjoint sets $\{\Sc_{i}\}\subset\M_{E}$, i.e. 
    $\exists$ disjoint sets $\{A_i\}$ such that $\Sc_i=A_i\cap E$, 
    then 
    $\cp{i}{\infty}\Sc_i=(\cp{i}{\infty}A_i)\cap E$. 
    As $\cp{i}{\infty}A_i\in\M$, 
    $\cp{i}{\infty}\Sc_i\in\M_{E}$.

    On the other hand, for $\Sc\in\M_{E}$, i.e. $\Sc=A\cap E$ 
    for $A\in\M$, it's clear that 
    $\Sc^{c}=A^{c}\cap E$ 
    (complement on $E$). 
    As $A^{c}\in\M$, $\Sc^{c}\in\M_{E}$. 
    So $(E,\M_{E})$ is a $\sigma$-algebra.

    Then, we show $\nu$ is well-defined. 
    If $\exists B\in\mathcal{M}$ s.t. $A\cap E=B\cap E$, 
    then by $(a)$, $\mu(A)=\mu(B)$. 
    So $\nu(A\cap E)=\mu(A)=\mu(B)$, which means $\nu(A\cap E)$ 
    is unrelated to the representation element $A$. 
    It means $\nu$ is well-defined.

    Finally, we show $\nu$ be a measure on $\M_{E}$. 
    For disjoint sets $\{\Sc_{i}\}\subset\M_{E}$, 
    i.e. $\Sc_{i}=A_{i}\cap E$ with $\{A_i\}$ disjoint, 
    \begin{displaymath}
        \nu(\cp{i}{\infty}\Sc_i)=
        \nu((\cp{i}{\infty}A_i)\cap E)
        =\mu(\cp{i}{\infty}A_i)
        =\sm{i}{\infty}\mu(A_i)
        =\sm{i}{\infty}\nu(\Sc_i).
    \end{displaymath}  
    And $\nu(\emptyset)=\mu(\emptyset)=0$, 
    $\nu(A\cap E)=\mu(A)\ge 0$. 
    So $\nu$ is a measure on $\M_{E}$.
\end{proof}
\begin{homeworkProblem}
    Complete the proof of Theorem 1.19.
\end{homeworkProblem}
\begin{proof}
    Consider the case $\mu(E)=\infty$. 
    Set $E_{n}:=\{x:n-1<|x|<n\}\cap E$, it's clear that 
    $E=(\mathbb{Z}\cap E)\cup(\cp{n}{\infty}E_{n})$ 
    and $\mu(E_{n})<\infty$.

    $(a)\Rightarrow(c)$: For each $E_{n}$, 
    $\exists$ $F_{\sigma}$-sets $K_{n}$ satisfies:
    \begin{itemize}
        \item $K_{n}\subset E_{n}$,
        \item $E_{n}=K_{n}\cup Q_{n}$ with $\mu(Q_{n})=0$. 
    \end{itemize}
    Then:
    \begin{displaymath}
        E=(\cp{n}{\infty}K_{n})\cup
        (\cp{n}{\infty}Q_{n})\cup(E\cap\mathbb{Z}).
    \end{displaymath}
    By the definition of $F_{\sigma}$ set, 
    $K:=\cp{n}{\infty}K_{n}$ is a $F_{\sigma}$ set. 
    And $\mu(\cp{n}{\infty}Q_{n}\cup(E\cap\mathbb{Z}))\le
    \mu(\cp{n}{\infty}Q_{n})+\mu(\mathbb{Z})=0$.
    It means $\exists$ $F_{\sigma}$-set $K$ and $\mu(\Sc)=0$ s.t. 
    $E=K\cup\Sc$.

    $(a)\Rightarrow(b)$: For each $E_{n}$, $\exists$ $G_{\delta}$-sets 
    $V_{n}$ satisfies:
    \begin{itemize}
        \item $E_{n}\subset V_{n}$.
        \item $V_{n}\subset \{x:n-1<|x|<n\}$ 
        (if not so, just choose $V:=V_{n}\cap\{x:n-1<|x|<n\}$) 
        to satisfy this condition. 
        \item $E_{n}=V_{n}\setminus P_{n}$ with $\mu(P_{n})=0$. 
    \end{itemize}
    Then 
    \begin{displaymath}
        E=(\cp{n}{\infty}V_{n})\setminus(\cp{n}{\infty}P_{n})
        \cup(E\cap\mathbb{Z}).
    \end{displaymath}
    First, $\mu(\cp{n}{\infty}P_{n})=0$. 
    Then, we show $\cp{n}{\infty}V_{n}$ is a $G_{\delta}$-set.

    As $V_{n}$ are $G_{\delta}$-sets, 
    $\exists$ open sets $G_{n,j}\subset\{x:n-1<|x|<n\}$ s.t. 
    $V_{n}\subset G_{n,j}$ and 
    $\cap_{j=1}^{\infty}G_{n,j}=V_{n}$. 
    Now 
    \begin{displaymath}
        \cp{n}{\infty}V_{n}=\cp{n}{\infty}\cap_{j=1}^{\infty}G_{n,j}
        =\cap_{j=1}^{\infty}\cp{n}{\infty}G_{n,j}.
    \end{displaymath}
    The second equation holds for the fact that $\forall n_{1}\neq n_{2}$, 
    $j_{1},j_{2}\in\mathbb{N}$, 
    $G_{n_{1},j_{1}}\cap G_{n_2,j_2}=\emptyset$.
    As $\cp{n}{\infty}G_{n,j}$ is open, $\cp{n}{\infty}V_{n}$ is a 
    $G_{\delta}$-set,
    i.e. $\exists$ open sets $O_{i}$ 
    s.t. $\cp{n}{\infty}V_{n}=\cap_{i=1}^{\infty}O_{i}$.

    Then, for $\mu(E\cap \mathbb{Z})=0$, 
    $\exists$ a sequence of open sets $U_{n}$ s.t. 
    $\mu(U_{n})\le\frac{1}{n}$ and $E\cap Z\subset U_{n}$, i.e. 
    $\cp{n}{\infty}V_{n}\cup(E\cap \mathbb{Z})
    =\cap_{i=1}^{\infty}(O_{i}\cap U_{i})$, it is a $G_{\delta}$ 
    set. 
    So the result holds.
\end{proof}
\end{document}