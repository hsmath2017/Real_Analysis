\section{Introduction}
\begin{exm}
    The \textbf{length} of an interval 
    $[a,b]\subset\mathbb{R}$ is $b-a$.
\end{exm}
\begin{exm}
    The \textbf{area} of a rectangle 
    $[a_1,b_1]\times [a_2,b_{2}]\subset\mathbb{R}^{2}$ 
    is $(b_1-a_1)(b_2-a_2)$.
\end{exm}
\begin{exm}
    The \textbf{volume} of a cube 
    $[a_1,b_1]\times [a_2,b_2]\times [a_3,b_3]\subset\mathbb{R}^3$ 
    is $(b_1-a_1)(b_2-a_2)(b_3-a_3)$.
\end{exm}
\begin{rem}
    The length, area and volume have something in common. 
    They are all derived by the \textbf{size} of 
    a subset $\Sc\subset\mathbb{R}^{n}$. 
    Then, for $X=\mathbb{R}^n$, we can give the `volume' 
    of a subset $\Sc\subset X$ with some necessary properties.  
\end{rem}
\begin{ntn}
    For a set $\Sc$, we denote 
    \begin{displaymath}
        \mathcal{P}(\Sc):=\{\M:\M\subset\Sc\}.
    \end{displaymath}
\end{ntn}
\begin{defn}
    \label{Defn:volume}
    In $\mathbb{R}^{n}$, the \textit{volume} is a function 
    $\mu:\mathcal{D}\subset 
    \mathcal{P}(\mathbb{R}^{n})\rightarrow\mathbb{R}$ satisfies:
    \begin{itemize}
        \item For a sequence $\{E_{i}\}_1^{\infty}\subset\mathcal{D}$ 
        and $\forall i\neq j$, $E_i\cap E_{j}=\emptyset$, 
        $\mu(\cp{i}{\infty}E_{i})=\sm{i}{\infty}\mu(E_i)$.
        \item If $E\subset F$, then $\mu(E)\le\mu(F)$.
        \item If there exists an isometric transformation 
        $O:\mathcal{D}\rightarrow\mathcal{D}$ such that $O(E)=F$, 
        then $\mu(E)=\mu(F)$.
        \item $\mu([0,1]^{n})=1$.
    \end{itemize}
\end{defn}
\begin{thm}
    \label{Thm:NotMeasurableSet}
    $\mathcal{D}\subsetneq\mathcal{P}(\mathbb{R}^{n})$.
\end{thm}
\begin{proof}
    We show the case $n=1$. 
    On $[0,1]$, we choose the relation $\sim$ as follows:
    \begin{displaymath}
        x\sim y\Leftrightarrow x-y\in\mathbb{Q},
    \end{displaymath}
    then define $E:=[0,1]/\sim$, and $E_{t}:=\{x:x-t\in E\}$, 
    denote $\tilde{E}:=\cup_{t\in\mathbb{Q}\cap[-1,1]}E_{t}$, 
    it's clear that 
    \begin{displaymath}
        [0,1]\subset\tilde{E}\subset[-1,2],
    \end{displaymath}
    by the second property, $1\le\mu(\tilde{E})\le 3$.

    By the constuction of $E$, $\forall t_{1}\neq t_2$, 
    $E_{t_1}\cap E_{t_2}=\empty$. 
    Since $E_{t_1}$ can be transformed into $E_{t_2}$ by translation, 
    by the third property, $\mu(E_{t_1})=\mu(E_{t_3})$.

    Then, by the first property, we have 
    \begin{displaymath}
        1\le\aleph_{0}\mu(E)\le 3.
    \end{displaymath}
    It's absurd. So we can't define the volume of $E$.
\end{proof}
\begin{rem}
    By Theorem \ref{Thm:NotMeasurableSet}, 
    we should discuss the properties of $\mathcal{D}$. 
    It's the motivation of introducing 
    $\sigma$-algebras.
\end{rem}
\section{$\sigma-$algebras}
\begin{defn}
    \label{Defn:SigmaAlg}
    Given a set $X$, $\mathcal{A}\subset\mathcal{P}(X)$ 
    is an \textit{algebra} if 
    \begin{itemize}
        \item $\mathcal{A}\neq\emptyset$.
        \item If $E_1,E_2\in\mathcal{A}$, $E_1\cup E_2\in\mathcal{A}$.
        \item If $E\in\mathcal{A}$, $E^{c}\in\mathcal{A}$.
    \end{itemize}
    $\mathcal{A}$ is a \textit{$\sigma$-algebra} if 
    $\mathcal{A}$ is an algebra and 
    $\{E_{k}\}_1^{\infty}\subset\mathcal{A}$ yields 
    $\cp{k}{\infty}E_k\in\mathcal{A}$.
\end{defn}
\begin{rem}
    By Definition \ref{Defn:SigmaAlg}, an algebra is closed 
    under finite union and supplementation, 
    and an $\sigma$-algebra is closed under 
    countable union and supplementation.
\end{rem}
\begin{exc}
    If $\mathcal{A}\subset\mathcal{P}(X)$ is an algebra, 
    show $\emptyset,X\in\mathcal{A}$.
\end{exc}
\begin{exc}
    If $\mathcal{A}$ is a $\sigma-$algebra and 
    $\{E_{k}\}_{1}^{\infty}\subset\mathcal{A}$, show that 
    $\cap_{k=1}^{\infty}E_{k}\in\mathcal{A}$.
\end{exc}
\begin{lem}
    \label{Lem:IntersecOfSigmaAlg}
    The intersection of any family of $\sigma$-algebras is a 
    $\sigma$-algebra.
\end{lem}
\begin{exc}
    Prove Lemma \ref{Lem:IntersecOfSigmaAlg}.
\end{exc}
\begin{defn}
\label{Defn:GeneratedSigmaAlg}
For $\mathcal{E}\subset\mathcal{P}(X)$, $\mathcal{M}(\mathcal{E})$ 
is called the \textit{$\sigma$-algebra generated by $\mathcal{E}$} 
if $\mathcal{E}$ is the intersection of all $\sigma$-algebras 
containing $\mathcal{E}$
\end{defn}
\begin{rem}
    $\mathcal{M}(\mathcal{E})$ is the smalled $\sigma$-algebra 
    containing $\mathcal{E}$.
\end{rem}
\begin{lem}
    \label{Lem:GeneratedSigmaAlgLem}
    If $\mathcal{E}\subset\mathcal{M}(\mathcal{F})$, 
    then $\M(\mathcal{E})\subset\M(\mathcal{F})$.
\end{lem}
\begin{exc}
    Prove Lemma \ref{Lem:GeneratedSigmaAlgLem}.
\end{exc}
\begin{defn}
    If $(X,\tau)$ is a topological space, then 
    the \textit{Borel $\sigma$-algebra} $\mathcal{B}_{X}$ is the 
    $\sigma$-algebra generated by the family of open sets 
    in $X$. 
    And the sets in $\mathcal{B}_{X}$ is called \textit{Borel sets}.
\end{defn}
\begin{exc}
    $\mathcal{B}_{\mathbb{R}}$ is generated by each of the following:
    \begin{itemize}
        \item $\mathcal{E}_{1}:=\{(a,b):a<b\}$,
        \item $\mathcal{E}_{2}:=\{(a,b]:a<b\}$,
        \item $\mathcal{E}_{3}:=\{[a,b):a<b\}$,
        \item $\mathcal{E}_{4}:=\{[a,b]:a<b\}$.
    \end{itemize}
\end{exc}
\begin{defn}
    \label{Defn:ProdSigmaAlg}
    Let $\{X_{\alpha}\}_{\alpha\in A}$ be an indexed collection 
    of nonempty sets, $X:=\prod_{\alpha\in A}X_{\alpha}$, 
    $\M_{\alpha}$ is a $\sigma$-algebra on $X_{\alpha}$, 
    and $\pi_{\alpha}:X\rightarrow X_{\alpha}$ is a projection, 
    the \textit{product $\sigma$-algebra} 
    \begin{displaymath}
        \otimes_{\alpha\in A}\M_{\alpha}:=\M\left(
            \left\{\pi_{\alpha}^{-1}(E_{\alpha}):
            E_{\alpha}\in\M_{\alpha},\alpha\in A\right\}
        \right).
    \end{displaymath}
\end{defn}
\begin{rem}
    The product $\sigma$-algebra is the 
    $\sigma$-algebra generated by a series of 
    $\sigma$-algebras.
\end{rem}
\begin{prop}
    \label{Prop:GeneraterOfProdAlg}
    $\otimes_{k=1}^{\infty}\M_{k}$ is generated by 
    $\{\prod_{k=1}^{\infty}E_{k}:E_{k}\subset\M_{k}\}$.
\end{prop}
\begin{proof}
    On one hand, 
    \begin{displaymath}
    \pi_{k}^{-1}(E_{k})=(X_{1},\ldots,X_{k-1},E_{k},X_{k+1},\ldots), 
    \end{displaymath}
    on the other hand, 
    \begin{displaymath}
        \prod_{k=1}^{\infty}E_{k}=\cap_{k=1}^{\infty}\pi_{k}^{-1}(E_k),
    \end{displaymath}
    then the 
    result follows from 
    Lemma \ref{Lem:GeneratedSigmaAlgLem}.
\end{proof}
\begin{exc}
    Show that $\mathcal{B}_{\mathbb{R}^n}
    =\otimes_{1}^{n}\mathcal{B}_{\mathbb{R}}$.
\end{exc}
\section{Measures}
\begin{defn}[Measure]
    \label{Defn:Measure}
    A \textit{measure} on $(X,\M)$ is 
    a function $\mu:\M\rightarrow[0,\infty]$ such that 
    \begin{itemize}
        \item $\mu(\emptyset)=0$,
        \item If $\{E_{j}\}_{1}^{\infty}$ pairwise disjoint, 
        $\mu(\cp{j}{\infty}E_j)=\sm{j}{\infty}\mu(E_j)$.
    \end{itemize}
\end{defn}
\begin{rem}
    Definition \ref{Defn:Measure} is a 
    generalization of Definition \ref{Defn:volume}.
\end{rem}
\begin{rem}
    $\M$ is a $\sigma$-algebra, so $\mu(\cp{j}{\infty}E_{j})$ 
    is well-defined. 
    It means that $\M$ \textbf{must} be a $\sigma$-algebra.
\end{rem}
\begin{defn}
    \label{Defn:MeasureWithFinite}
    $(X,\M)$ is a \textit{measurable space}, the sets in $\M$ are 
    \textit{measurable sets}, and $(X,\M,\mu)$ is a 
    \textit{measurable space}.
    \begin{itemize}
        \item If $\mu(X)<\infty$, $\mu$ is \textit{finite}.
        \item If $X=\cp{j}{\infty}E_{j}$, $E_{j}\in\M$ and 
        $\forall j$, $\mu(E_j)<\infty$, $\mu$ is 
        \textit{$\sigma$-finite}.
        \item If $\forall E\in\M$, $\mu(E)=\infty$, 
        $\exists F\subset E$ s.t. 
        $0<\mu(F)<\infty$, then $\mu$ is \textit{semi-finite}.
    \end{itemize}
\end{defn}
\begin{rem}
    If $\mu$ is volume in $\mathbb{R}^{n}$, then $\mu(X)=\infty$ 
    means $X$ is unbounded. 
    So, it's essential to show a measure is 
    finite or not.
\end{rem}
\begin{exm}
    \label{Exm:CountingAndDirac}
    Given a function $f:X\rightarrow [0,\infty)$, 
    for $E\in\mathcal{P}(X)$, 
    \begin{displaymath}
        \mu_{f}(E):=\sum_{x\in E}f(x)
    \end{displaymath}
    is a measure on $X$. 
    If $f\equiv 1$, $\mu$ is called \textit{counting measure}. 
    If $f(x)=\left\{\begin{array}{rl}
        1,x=x_0,\\
        0,x\neq x_0,\\
    \end{array}\right.$
    then $\mu$ is called \textit{Dirac delta measure} on $x_{0}$, 
    marked as $\delta_{x_0}$.
\end{exm}
\begin{exc}
    Exclaim the meanings of 
    counting measure and Dirac delta measure 
    by some examples in real world.
\end{exc}
\section{Outer Measures}
\section{Lebesgue Measure}
