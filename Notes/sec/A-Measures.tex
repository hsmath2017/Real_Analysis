\section{Introduction}
\begin{exm}
    The \textbf{length} of an interval 
    $[a,b]\subset\mathbb{R}$ is $b-a$.
\end{exm}
\begin{exm}
    The \textbf{area} of a rectangle 
    $[a_1,b_1]\times [a_2,b_{2}]\subset\mathbb{R}^{2}$ 
    is $(b_1-a_1)(b_2-a_2)$.
\end{exm}
\begin{exm}
    The \textbf{volume} of a cube 
    $[a_1,b_1]\times [a_2,b_2]\times [a_3,b_3]\subset\mathbb{R}^3$ 
    is $(b_1-a_1)(b_2-a_2)(b_3-a_3)$.
\end{exm}
\begin{rem}
    The length, area and volume have something in common. 
    They are all derived by the \textbf{size} of 
    a subset $\Sc\subset\mathbb{R}^{n}$. 
    Then, for $X=\mathbb{R}^n$, we can give the `volume' 
    of a subset $\Sc\subset X$ with some necessary properties.  
\end{rem}
\begin{ntn}
    For a set $\Sc$, we denote 
    \begin{displaymath}
        \mathcal{P}(\Sc):=\{\M:\M\subset\Sc\}.
    \end{displaymath}
\end{ntn}
\begin{defn}
    \label{Defn:volume}
    In $\mathbb{R}^{n}$, the \textit{volume} is a function 
    $\mu:\mathcal{D}\subset 
    \mathcal{P}(\mathbb{R}^{n})\rightarrow\mathbb{R}$ satisfies:
    \begin{itemize}
        \item For a sequence $\{E_{i}\}_1^{\infty}\subset\mathcal{D}$ 
        and $\forall i\neq j$, $E_i\cap E_{j}=\emptyset$, 
        $\mu(\cp{i}{\infty}E_{i})=\sm{i}{\infty}\mu(E_i)$.
        \item If $E\subset F$, then $\mu(E)\le\mu(F)$.
        \item If there exists an isometric transformation 
        $O:\mathcal{D}\rightarrow\mathcal{D}$ such that $O(E)=F$, 
        then $\mu(E)=\mu(F)$.
        \item $\mu([0,1]^{n})=1$.
    \end{itemize}
\end{defn}
\begin{thm}
    \label{Thm:NotMeasurableSet}
    $\mathcal{D}\subsetneq\mathcal{P}(\mathbb{R}^{n})$.
\end{thm}
\begin{proof}
    We show the case $n=1$. 
    On $[0,1]$, we choose the relation $\sim$ as follows:
    \begin{displaymath}
        x\sim y\Leftrightarrow x-y\in\mathbb{Q},
    \end{displaymath}
    then define $E:=[0,1]/\sim$, and $E_{t}:=\{x:x-t\in E\}$, 
    denote $\tilde{E}:=\cup_{t\in\mathbb{Q}\cap[-1,1]}E_{t}$, 
    it's clear that 
    \begin{displaymath}
        [0,1]\subset\tilde{E}\subset[-1,2],
    \end{displaymath}
    by the second property, $1\le\mu(\tilde{E})\le 3$.

    By the constuction of $E$, $\forall t_{1}\neq t_2$, 
    $E_{t_1}\cap E_{t_2}=\emptyset$. 
    Since $E_{t_1}$ can be transformed into $E_{t_2}$ by translation, 
    by the third property, $\mu(E_{t_1})=\mu(E_{t_3})$.

    Then, by the first property, we have 
    \begin{displaymath}
        1\le\aleph_{0}\mu(E)\le 3.
    \end{displaymath}
    It's absurd. So we can't define the volume of $E$.
\end{proof}
\begin{rem}
    By Theorem \ref{Thm:NotMeasurableSet}, 
    we should discuss the properties of $\mathcal{D}$. 
    It's the motivation of introducing 
    $\sigma$-algebras.
\end{rem}
\section{$\sigma-$algebras}
\begin{defn}
    \label{Defn:SigmaAlg}
    Given a set $X$, $\mathcal{A}\subset\mathcal{P}(X)$ 
    is an \textit{algebra} if 
    \begin{itemize}
        \item $\mathcal{A}\neq\emptyset$.
        \item If $E_1,E_2\in\mathcal{A}$, $E_1\cup E_2\in\mathcal{A}$.
        \item If $E\in\mathcal{A}$, $E^{c}\in\mathcal{A}$.
    \end{itemize}
    $\mathcal{A}$ is a \textit{$\sigma$-algebra} if 
    $\mathcal{A}$ is an algebra and 
    $\{E_{k}\}_1^{\infty}\subset\mathcal{A}$ yields 
    $\cp{k}{\infty}E_k\in\mathcal{A}$.
\end{defn}
\begin{rem}
    By Definition \ref{Defn:SigmaAlg}, an algebra is closed 
    under finite union and supplementation, 
    and an $\sigma$-algebra is closed under 
    countable union and supplementation.
\end{rem}
\begin{exc}
    If $\mathcal{A}\subset\mathcal{P}(X)$ is an algebra, 
    show $\emptyset,X\in\mathcal{A}$.
\end{exc}
\begin{proof}
    If $\mathcal{A}$ is an algebra, $\mathcal{A}\neq\emptyset$. 
    If $E\in\mathcal{A}$, we have $\emptyset=E\cap E^c\in\mathcal{A}$, 
    $X=E\cup E^c\in\mathcal{A}$.
\end{proof}
\begin{exc}
    If $\mathcal{A}$ is a $\sigma-$algebra and 
    $\{E_{k}\}_{1}^{\infty}\subset\mathcal{A}$, show that 
    $\cap_{k=1}^{\infty}E_{k}\in\mathcal{A}$.
\end{exc}
\begin{proof}
    If $\mathcal{A}$ is a $\sigma-$algebra and 
    $\{E_{k}\}_{1}^{\infty}\subset\mathcal{A}$, 
    $\{E_{k}^c\}_{1}^{\infty}\subset\mathcal{A}$,
    which implies that $\cup_{k=1}^{\infty}E_{k}^c\in\mathcal{A}$.
    We can derive that $\cap_{k=1}^{\infty}E_{k}=(\cup_{k=1}^{\infty}E_{k}^c)^c\in\mathcal{A}$.
\end{proof}
\begin{lem}
    \label{Lem:IntersecOfSigmaAlg}
    The intersection of any family of $\sigma$-algebras is a 
    $\sigma$-algebra.
\end{lem}
\begin{exc}
    Prove Lemma \ref{Lem:IntersecOfSigmaAlg}.
\end{exc}
\begin{proof}
    If $\mathcal{A}_{i}\subset\mathcal{P}(X)$, $i\in I$, 
    where $I$ is index set. We can notice that $\cap_{i \in I}\mathcal{A}_{i}\neq\emptyset$,
    since $\{\emptyset,X\}\subset\cap_{i \in I}\mathcal{A}_{i}$.
    
    If $E\in\cap_{i\in I}\mathcal{A}_{i}$ 
    $\Rightarrow$ $\forall i\in I$, 
    $E\in\mathcal{A}_{i}$. 
    By the third property of $\sigma$-algebras, $\forall i\in I$, 
    $E^c\in\mathcal{A}_{i}$, 
    which implies $E^c\in\cap_{i \in I}\mathcal{A}_{i}$.
    
    If $\{E_{k}\}_1^{\infty}\subset\cap_{i \in I}\mathcal{A}_{i}$ 
    $\Rightarrow$ 
    $\forall i\in I$, $\{E_{k}\}_1^{\infty}\in\mathcal{A}_{i}$. 
    By the second property of $\sigma$-algebras, $\forall i\in I$, 
    $\cp{k}{\infty}E_{k}\in\mathcal{A}_{i}$, 
    which implies $\cp{k}{\infty}E_k\in\cap_{i\in I}\mathcal{A}_{i}$.
\end{proof}
\begin{defn}
\label{Defn:GeneratedSigmaAlg}
For $\mathcal{E}\subset\mathcal{P}(X)$, $\mathcal{M}(\mathcal{E})$ 
is called the \textit{$\sigma$-algebra generated by $\mathcal{E}$} 
if $\mathcal{M}(\mathcal{E})$ is the intersection of all $\sigma$-algebras 
containing $\mathcal{E}$
\end{defn}
\begin{rem}
    $\mathcal{M}(\mathcal{E})$ is the smalled $\sigma$-algebra 
    containing $\mathcal{E}$.
\end{rem}
\begin{lem}
    \label{Lem:GeneratedSigmaAlgLem}
    If $\mathcal{E}\subset\mathcal{M}(\mathcal{F})$, 
    then $\M(\mathcal{E})\subset\M(\mathcal{F})$.
\end{lem}
\begin{exc}
    Prove Lemma \ref{Lem:GeneratedSigmaAlgLem}.
\end{exc}
\begin{proof}
    $\mathcal{M}(\mathcal{F})$ is a $\sigma$-algebra containing $\mathcal{E}$; it therefore contains $\mathcal{M}(\mathcal{E})$.
\end{proof}
\begin{defn}
    If $(X,\tau)$ is a topological space, then 
    the \textit{Borel $\sigma$-algebra} $\mathcal{B}_{X}$ is the 
    $\sigma$-algebra generated by the family of open sets 
    in $X$. 
    And the sets in $\mathcal{B}_{X}$ is called \textit{Borel sets}.
\end{defn}
\begin{exc}
    \label{Exer:BRGeneratedsets}
    $\mathcal{B}_{\mathbb{R}}$ is generated by each of the following:
    \begin{itemize}
        \item $\mathcal{E}_{1}:=\{(a,b):a<b\}$,
        \item $\mathcal{E}_{2}:=\{(a,b]:a<b\}$,
        \item $\mathcal{E}_{3}:=\{[a,b):a<b\}$,
        \item $\mathcal{E}_{4}:=\{[a,b]:a<b\}$.
    \end{itemize}
\end{exc}
\begin{proof}
    The elements of $\mathcal{E}_{j}$ for $j\neq3,4$ are open or closed,
    and the elements of $\mathcal{E}_{3}$ and $\mathcal{E}_{4}$ can be expressed
    by a countable intersection of open sets, for example, $(a,b]=\cap_{1}^{\infty}(a,b+n^{-1})$.
    All of these are Borel sets, so by Lemma \ref{Lem:GeneratedSigmaAlgLem},
    $\mathcal{M}(\mathcal{E}_{j})\subset\mathcal{B}_{\mathbb{R}}$ for all $j$.On the other hand,
    every open set in $\mathbb{R}$ is a countable union of open intervals,
    so by Lemma \ref{Lem:GeneratedSigmaAlgLem} again, $\mathcal{B}_{\mathbb{R}}\subset\mathcal{M}(\mathcal{E}_{1})$. 
    That $\mathcal{M}(\mathcal{E}_{j})\subset\mathcal{B}_{\mathbb{R}}$ for $j\geq2$ can be established
    by showing that all open intervals lie in $\mathcal{M}(\mathcal{E}_{j})$ and applying Lemma \ref{Lem:GeneratedSigmaAlgLem}.
    For example, $(a,b)=\cup_{1}^{\infty}(a,b-n^{-1}]\in\mathcal{M}(\mathcal{E}_{2})$.
\end{proof}
\begin{defn}
    \label{Defn:ProdSigmaAlg}
    Let $\{X_{\alpha}\}_{\alpha\in A}$ be an indexed collection 
    of nonempty sets, $X:=\prod_{\alpha\in A}X_{\alpha}$, 
    $\M_{\alpha}$ is a $\sigma$-algebra on $X_{\alpha}$, 
    and $\pi_{\alpha}:X\rightarrow X_{\alpha}$ is a projection, 
    the \textit{product $\sigma$-algebra} 
    \begin{displaymath}
        \otimes_{\alpha\in A}\M_{\alpha}:=\M\left(
            \left\{\pi_{\alpha}^{-1}(E_{\alpha}):
            E_{\alpha}\in\M_{\alpha},\alpha\in A\right\}
        \right).
    \end{displaymath}
\end{defn}
\begin{rem}
    The product $\sigma$-algebra is the 
    $\sigma$-algebra generated by a series of 
    $\sigma$-algebras.
\end{rem}
\begin{prop}
    \label{Prop:GeneraterOfProdAlg}
    $\otimes_{k=1}^{\infty}\M_{k}$ is generated by 
    $\{\prod_{k=1}^{\infty}E_{k}:E_{k}\subset\M_{k}\}$.
\end{prop}
\begin{proof}
    On one hand, 
    \begin{displaymath}
    \pi_{k}^{-1}(E_{k})=(X_{1},\ldots,X_{k-1},E_{k},X_{k+1},\ldots), 
    \end{displaymath}
    on the other hand, 
    \begin{displaymath}
        \prod_{k=1}^{\infty}E_{k}=\cap_{k=1}^{\infty}\pi_{k}^{-1}(E_k),
    \end{displaymath}
    then the 
    result follows from 
    Lemma \ref{Lem:GeneratedSigmaAlgLem}.
\end{proof}
\begin{exc}
    Show that $\mathcal{B}_{\mathbb{R}^n}
    =\otimes_{1}^{n}\mathcal{B}_{\mathbb{R}}$.
\end{exc}
\begin{proof}
    $\mathcal{B}_{\mathbb{R}^n}$ is generated by  $\{\prod_{k=1}^{n}(a_k,b_k):a_k<b_k \quad and \quad a_k,b_k\in\mathbb{R}\}$.
    By Proposition \ref{Prop:GeneraterOfProdAlg} and Exercise \ref{Exer:BRGeneratedsets},
    $\otimes_{1}^{n}\mathcal{B}_{\mathbb{R}}$ is generated by $\{\prod_{k=1}^{n}(a_k,b_k):a_k<b_k \quad and \quad a_k,b_k\in\mathbb{R}\}$ as well,
    so $\mathcal{B}_{\mathbb{R}^n}=\otimes_{1}^{n}\mathcal{B}_{\mathbb{R}}$.
\end{proof}
\section{Measures}
\begin{defn}[Measure]
    \label{Defn:Measure}
    A \textit{measure} on $(X,\M)$ is 
    a function $\mu:\M\rightarrow[0,\infty]$ such that 
    \begin{itemize}
        \item $\mu(\emptyset)=0$,
        \item If $\{E_{j}\}_{1}^{\infty}$ pairwise disjoint, 
        $\mu(\cp{j}{\infty}E_j)=\sm{j}{\infty}\mu(E_j)$.
    \end{itemize}
\end{defn}
\begin{rem}
    Definition \ref{Defn:Measure} is a 
    generalization of Definition \ref{Defn:volume}.
\end{rem}
\begin{rem}
    $\M$ is a $\sigma$-algebra, so $\mu(\cp{j}{\infty}E_{j})$ 
    is well-defined. 
    It means that $\M$ \textbf{must} be a $\sigma$-algebra.
\end{rem}
\begin{defn}
    \label{Defn:MeasureWithFinite}
    $(X,\M)$ is a \textit{measurable space}, the sets in $\M$ are 
    \textit{measurable sets}, and $(X,\M,\mu)$ is a 
    \textit{measurable space}.
    \begin{itemize}
        \item If $\mu(X)<\infty$, $\mu$ is \textit{finite}.
        \item If $X=\cp{j}{\infty}E_{j}$, $E_{j}\in\M$ and 
        $\forall j$, $\mu(E_j)<\infty$, $\mu$ is 
        \textit{$\sigma$-finite}.
        \item If $\forall E\in\M$, $\mu(E)=\infty$, 
        $\exists F\subset E$ s.t. 
        $0<\mu(F)<\infty$, then $\mu$ is \textit{semi-finite}.
    \end{itemize}
\end{defn}
\begin{rem}
    If $\mu$ is volume in $\mathbb{R}^{n}$, then $\mu(X)=\infty$ 
    means $X$ is unbounded. 
    So, it's essential to show a measure is 
    finite or not.
\end{rem}
\begin{exm}
    \label{Exm:CountingAndDirac}
    Given a function $f:X\rightarrow [0,\infty)$, 
    for $E\in\mathcal{P}(X)$, 
    \begin{displaymath}
        \mu_{f}(E):=\sum_{x\in E}f(x)
    \end{displaymath}
    is a measure on $X$. 
    If $f\equiv 1$, $\mu$ is called \textit{counting measure}. 
    If $f(x)=\left\{\begin{array}{rl}
        1,x=x_0,\\
        0,x\neq x_0,\\
    \end{array}\right.$
    then $\mu$ is called \textit{Dirac delta measure} on $x_{0}$, 
    marked as $\delta_{x_0}$.
\end{exm}
\begin{exc}
    Exclaim the meanings of 
    counting measure and Dirac delta measure 
    by some examples in real world.
\end{exc}
\begin{proof}
    Counting measure is used to describe the number of elements in a set. 
    Nevertheless, Dirac delta measure is a generalized function that is concentrated at a single point.
    For instance, if there are 5 apples and 3 oranges, the counting measure of the set “fruits in the basket” would be 8.
    For the Dirac delta measure,Think about a ruler. The Dirac delta measure at a specific point on the ruler would 
    represent the mass or density concentrated at that point. 
\end{proof}
\begin{thm}
    \label{Thm:PropertiesOfMeasure}
    Given a measurable space $(X,\M,\mu)$, 
    \begin{enumerate}[(a)]
        \item \textit{Monotonicity:} If $E,F\in\M$ and $E\subset F$, 
        then $\mu(E)\le\mu(F)$. 
        \item \textit{Subadditivity:} 
        If $\{E_{j}\}_{1}^{\infty}\subset\M$, 
        then $\mu\left(\cp{j}{\infty}E_j\right)\le\sm{j}{\infty}\mu(E_j)$.
        \item \textit{Continuity from below:} 
        If $\{E_j\}_{1}^{\infty}\subset\M$, 
        $E_1\subset E_2\subset\ldots$, then 
        $\mu(\cp{i}{\infty}E_{i})=\lim_{n\rightarrow\infty}\mu(E_n)$.
        \item \textit{Continuity from above:}
        If $\{E_j\}_{1}^{\infty}\subset\M$, 
        $E_1\supset E_2\supset\ldots$ and $\mu(E_{1})<\infty$, 
        then 
        $\mu(\cap_{i=1}^{\infty}E_{i})=\lim_{n\rightarrow\infty}\mu(E_n)$.
    \end{enumerate} 
\end{thm}
\begin{proof}
    (a) Since $F=E\cup (F\setminus E)$, 
    by Definition \ref{Defn:Measure}, 
    \begin{displaymath}
        \mu(F)=\mu(E)+\mu(F\setminus E)\ge\mu(E).
    \end{displaymath}
    (b) Mark 
    \begin{displaymath}
        F_{1}:=E_{1},\quad F_{k}:=E_{k}\setminus\cp{i}{k-1}E_{i},
    \end{displaymath}
    then $\{F_{k}\}$ are pairwise disjoint sets, and 
    $\cp{i}{\infty}E_{i}=\cp{i}{\infty}F_{i}$.  
    Then by Definition \ref{Defn:Measure}, 
    \begin{equation}
        \label{Equ:SubAddi1}
        \mu\left(\cp{i}{\infty}E_i\right)
        =\mu\left(\cp{i}{\infty}F_{i}\right)=\sm{i}{\infty}\mu(F_i).
    \end{equation}
    By the definition of $F_{k}$, 
    $\forall k\ge 1$, $F_{k}\subset E_{k}$, so 
    \begin{equation}
        \label{Equ:SubAddi2}
        \sm{i}{\infty}\mu(F_i)\le\sm{i}{\infty}(E_i).
    \end{equation}
    By \eqref{Equ:SubAddi1} and \eqref{Equ:SubAddi2}, 
    (b) is true. 

    (c) We set $E_0:=\emptyset$, since $\{E_{k}\}_{1}^{\infty}$ 
    increasing, $F_{k}:=E_{k}\setminus\left(\cp{i}{k-1}E_i\right)
    =E_{k}\setminus E_{k-1}$. 
    Then by Definition \ref{Defn:Measure}:
    \begin{displaymath}
        \begin{array}{rl}
        \mu(\cp{j}{\infty}E_j)=&\mu(\cp{j}{\infty}F_j)\\
        =&\sm{j}{\infty}\mu(F_j)\\
        =&\lim_{n\rightarrow\infty}\sm{j}{n}\mu(E_j\setminus E_{j-1})\\
        =&\lim_{n\rightarrow\infty}\mu(E_n).\\
        \end{array}
    \end{displaymath}

    (d) Mark $G_{j}:=E_{1}\setminus E_{j}$, 
    by Definition \ref{Defn:Measure} and $(c)$, 
    \begin{equation}
        \label{Equ:ContinuousFromAbove}
        \begin{array}{rl}
            \mu(E_1)&=\mu(\cap_{i=1}^{\infty}E_{i})
            +\lim_{j\rightarrow\infty}\mu(F_j)\\
            &=\mu(\cap_{i=1}^{\infty}E_{i})+\lim_{j\rightarrow\infty}
            (\mu(E_1)-\mu(E_{j})).
        \end{array}
    \end{equation} 
    Since $\mu(E_1)<\infty$, \eqref{Equ:ContinuousFromAbove} 
    means $\lim_{j\rightarrow\infty}\mu(E_{j})
    =\mu(\cap_{i=1}^{\infty}E_{i})$. 
\end{proof}
\begin{rem}
    The item (b) omit the condition 
    for $\{E_{k}\}$ disjoint. 
\end{rem}
\begin{rem}
    In this chapter, 
    unless otherwise specified, 
    we consider the measurable space 
    $(X,\M,\mu)$.
\end{rem}
\begin{exc}
    Give an example to show that 
    if $\mu(E_1)=\infty$, 
    (d) in Theorem \ref{Thm:PropertiesOfMeasure} may fail. 
\end{exc}
\begin{proof}
    For instance, consider the Lebesgue measure on the set of real numbers. 
    Let's assume $E_n=[n,+\infty)$,
    a sequence of intervals starting from $n$.
    These sets are decreasing, as 
    $E_1\supset E_2\supset\ldots$ and $\mu(E_n)=\infty$ for all $n$.
    Therefore, $\lim_{n\rightarrow\infty}\mu(E_n)=\infty$.
    However, $\cap_{i=1}^{\infty}E_{i}=\emptyset$, 
    $\mu(\cap_{i=1}^{\infty}E_{i})=0\neq\infty$.
    In this case, the theorem does not hold.
\end{proof}
\begin{defn}
    \label{Defn:NullSet}
    If $E\in\M$ satisfies $\mu(E)=0$, $E$ 
    is said to be a \textit{null set.}
\end{defn}
\begin{defn}
    \label{Defn:Trueae}
    If $P(x)$ is a statement which is true 
    for all $x$ outside of a null set $E$, 
    we say $P(x)$ is \textit{true a.e.}
\end{defn}
\begin{exc}
    If $\mu(E)=0$ and $F\subset E$, 
    can we conclude that $\mu(F)=0$? 
\end{exc}
\begin{proof}
    By Monotonicity in Theorem \ref{Thm:PropertiesOfMeasure},
    we can conclude that $\mu(F)=0$,
    since $0\leq\mu(F)\leq\mu(E)=0$.
    However, it may not be true that $F\in\M$. 
\end{proof}
\begin{defn}
    \label{Defn:CompleteMeas}
    If all subsets of any null sets are 
    in $\M$, we say $\mu$ is \textit{complete}.
\end{defn}
\begin{exm}
    The volume in Definition \ref{Defn:volume} 
    is a complete measure.
\end{exm}
\begin{thm}
    \label{Thm:CompletationForMeas}
    Let $\mathcal{N}:=\{N\in\M,\mu(N)=0\}$, 
    $\overline{\M}:=\{E\cup F:E\in\M,\; 
    F\subset N\text{ for some }N\in\mathcal{N}\}$, 
    then $\overline{\M}$ is a $\sigma$-algebra, 
    and there is a unique 
    extension $\bar{\mu}$ of $\mu$, 
    which is complete on $\overline{\M}$.
\end{thm}
\begin{exc}
    Prove Theorem \ref{Thm:CompletationForMeas} 
    (Hint: $\bar{\mu}(E\cup F):=\mu(E)$).
\end{exc}
\begin{proof}
    Since $\mathcal{N}$ and $\M$ are closed under countable unions,
    so is $\overline{\M}$.
    If $E\cup F\in\overline{\M}$ where $E\in\M$ and $F\subset N\in\mathcal{N}$,
    we can assume that $E\cap N=\emptyset$ 
    (otherwise, replace $F$ and $N$ by $F\setminus E$ and $N\setminus E$).
    Then $E\cup F=(E\cup N)\cap(N^c\cup F)$,
    so $(E\cup F)^c=(E\cup N)^c\cup(N\setminus F)$.
    But $(E\cup F)^c\in\M$ and $N\setminus F\subset N$, so that $(E\cup F)^c\in\overline{\M}$.
    Thus $\overline{\M}$ is a $\sigma$-algebra.

    If $E\cup F\in\overline{\M}$ as above, we set $\bar{\mu}(E\cup F):=\mu(E)$.
    This well defined, since if $E_1\cup F_1=E_2\cup F_2$ where $F_j\subset N_j$,
    then $E_1\subset E_2\cup N_2$ and so $\mu{E_1}\leq\mu{E_2}+\mu{N_2}=\mu{E_2}$,
    and likewise $\mu{E_2}\leq\mu{E_1}$.

    Firstly, we prove $\bar{\mu}$ is a measure. $\bar{\mu}(\emptyset)=\mu(\emptyset)=0$.
    If $\{E_j\cup F_j\}_{1}^{\infty}$ pairwise disjoint, 
    $\bar{\mu}(\cup_{j=1}^{\infty}(E_j\cup F_j))=\mu(\cup_{j=1}^{\infty}E_j)=\sum_{j=1}^{\infty}\mu{E_j}=\sum_{j=1}^{\infty}\bar{\mu}(E_j\cup F_j)$.
    Thus $\bar{\mu}$ is a measure.

    Secondly, we prove $\bar{\mu}$ is complete. If $\bar{\mu}(E)=0$, $E\in\mathcal{N}$.
    Any $F\subset E$, $\bar{\mu}(F)=\bar{\mu}(\emptyset\cup F)=\mu(\emptyset)=0$, 
    so $\bar{\mu}$ is complete.

    Finally, we prove $\bar{\mu}$ is the only complete measure on $\overline{\M}$ that extends $\mu$.
    If there exists another complete measure $\mu^*$ is a extension of $\mu$ on $\overline{\M}$.
    By monotonicity of $\mu^*$, $\bar{\mu}(E\cup F)=\mu(E)=\mu^*(E)\leq\mu^*(E\cup F)$.
    By completeness and subadditivity of $\mu^*$, 
    $\mu^*(E\cup F)\leq\mu^*(E)+\mu^*(F)=\mu^*(E)=\mu(E)=\bar{\mu}(E\cup F)$.
    We derive that for all $E\cup F\in\overline{\M}$, $\mu^*(E\cup F)=\bar{\mu}(E\cup F)$, 
    so $\mu^*=\bar{\mu}$.
\end{proof}
\section{Outer Measures}
\begin{rem}
    By Theorem \ref{Thm:NotMeasurableSet}, there exists 
    some sets $X\subset\mathbb{R}^{n}$ such that we can't define the 
    volume of $X$, but, on which condition can we define the volume of $X$?
    In this section, we define a map 
    $\mu^{*}:\mathcal{P}(X)\rightarrow [0,\infty]$, derive the 
    $\mu^{*}$-measurable sets $\M\subset \mathcal{P}(X)$, and 
    deduce a measure on $\M$. 
\end{rem}
\begin{defn}
    Given a set $X$, $\mu^{*}:\mathcal{P}(X)\rightarrow[0,\infty]$ 
    is an \textit{outer measure} if 
    \begin{itemize}
        \item $\mu^{*}(\emptyset)=0$,
        \item $\mu^{*}(A)\le\mu^{*}(B)$ if $A\subset B$, 
        \item $\mu^{*}(\cp{j}{\infty}A_{j})\le\sm{j}{\infty}\mu^{*}(A_{j})$.
    \end{itemize}
\end{defn}
\begin{rem}
    The outer measure only need 
    the subadditivity, rather than countable additivity. 
    So, we can derive an outer measure from the volume of unit cubes.
\end{rem}
\begin{prop}
    \label{Prop:OuterMeasureFromFunction}
    Given $\mathcal{E}\subset\mathcal{P}(X)$ and 
    $\rho:\mathcal{E}\rightarrow[0,\infty]$ such that 
    $\emptyset\in\mathcal{E}$, $x\in\mathcal{E}$ 
    and $\rho(\emptyset)=0$, for $A\subset X$, 
    \begin{equation}
        \label{Equ:DerivedOuterMeas}
        \mu^{*}(A):=\inf_{E_j\in\mathcal{E},A\subset\cp{j}{\infty}E_j}
        \sm{j}{\infty}\rho(E_j),
    \end{equation}
    then $\mu^{*}$ is an outer measure.
\end{prop}
\begin{rem}
    If $X=\mathbb{R}^{n}$, $\mathcal{E}$ marks all the 
    elementary cubes $\prod_{i=1}^{n}(a_i,b_i]$, 
    and 
    \begin{displaymath}
        \rho\left(\prod_{i=1}^{n}(a_i,b_i]\right)
        :=\prod_{i=1}^{n}(b_i-a_i),
    \end{displaymath}
    then 
    \eqref{Equ:DerivedOuterMeas} is the Lebesgue outer measure 
    on $\mathbb{R}^{n}$.
\end{rem}
\begin{exc}
    Prove Proposition \ref{Prop:OuterMeasureFromFunction}.
\end{exc}
Then, we should construct a measurable space 
$(X,\M,\mu)$ from the outer measure $\mu^{*}$. 
\begin{defn}
    \label{Defn:MuStarMeasurable}
    Given an outward measure $\mu^{*}$ on $\mathcal{P}(X)$, 
    a set $A\subset X$ is called \textit{$\mu^{*}$-measurable} 
    if 
    \begin{equation}
        \label{Equ:MeasurableCondition}
        \forall E\subset X,\quad \mu^{*}(E)=\mu^{*}(E\cap A)
        +\mu^{*}(E\cap A^c).
    \end{equation}
\end{defn}
\begin{thm}[Caratheodory]
    \label{Thm:CaratheodoryThm}
    Give an outer measure $\mu^{*}$ on $\mathcal{P}(X)$, 
    $\M:=\{\Sc\subset X:\Sc\text{ is }\mu^*-\text{measurable}\}$, 
    then $\M$ is a $\sigma$-algebra and 
    $\mu^{*}|_{\M}$ is a complete measure. 
\end{thm}
\begin{proof}
    We divide this proof into four steps.
    
    First, by equation \eqref{Equ:MeasurableCondition}, it's straightforward 
    that $\M$ is closed under complementation. 

    Second, we show that $\M$ is closed under finite union. 
    If $A,B\in\M$, by \eqref{Equ:MeasurableCondition}, 
    $\forall E\subset X$ we have:
    \begin{equation*}
        \label{Equ:ABMeasurable}
        \begin{array}{rl}
            \mu^{*}(E)&=\mu^{*}(E\cap A)+\mu^{*}(E\cap A^c),\\
            \mu^{*}(E)&=\mu^{*}(E\cap B)+\mu^{*}(E\cap B^c).\\
        \end{array}
    \end{equation*}
    Then:
    \begin{equation}
        \label{Equ:ExpressMuStarE}
        \begin{array}{rl}
        &\mu^{*}(E)=\mu^{*}(E\cap A)+\mu^{*}(E\cap A^c)\\
        =&\mu^{*}(E\cap A\cap B)+\mu^{*}(E\cap A\cap B^c)\\
        +&\mu^{*}(E\cap A^c\cap B)+\mu^{*}(E\cap A^c\cap B^c).
        \end{array}
    \end{equation}
    As $A\cup B=(A\cap B)\cup(A\cap B^c)\cup(A^c\cap B)$, 
    from the subadditivity, 
    \begin{equation}
        \label{Equ:ExpressionsForEcapAcupB}
        \begin{array}{rl}
        &\mu^{*}(E\cap A\cap B)+\mu^{*}(E\cap A\cap B^c)\\
        +&\mu^{*}(E\cap A^c\cap B)\ge\mu^{*}(E\cap(A\cup B)).\\
        \end{array}
    \end{equation}
    From \eqref{Equ:ExpressMuStarE} and 
    \eqref{Equ:ExpressionsForEcapAcupB}, 
    \begin{displaymath}
        \mu^{*}(E)\ge\mu^{*}(E\cap (A\cup B))+\mu^{*}(E\cup (A\cup B)).
    \end{displaymath}
    And from the subadditivity, we derive $A\cup B\in\M$, 
    which means $\M$ is an algebra. If $A\cap B=\emptyset$, 
    \begin{displaymath}
        \mu^{*}(A\cup B)=\mu^{*}(A\cup B\cap B)+
        \mu^{*}(A\cup B\cap B^c)=\mu^{*}(A)+\mu^{*}(B).
    \end{displaymath}
\end{proof}
\section{Lebesgue Measure}
