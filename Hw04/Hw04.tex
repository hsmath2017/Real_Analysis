\documentclass{article}

\usepackage{fancyhdr}
\usepackage{extramarks}
\usepackage{amsmath}
\usepackage{amsthm}
\newtheorem{lemma}{Lemma}
\usepackage{amsfonts}
\usepackage{tikz}
\usepackage[plain]{algorithm}
\usepackage{algpseudocode}

\usetikzlibrary{automata,positioning}

%
% Basic Document Settings
%

\topmargin=-0.45in
\evensidemargin=0in
\oddsidemargin=0in
\textwidth=6.5in
\textheight=9.0in
\headsep=0.25in

\linespread{1.1}

\pagestyle{fancy}
\lhead{\hmwkAuthorName}
\chead{\hmwkClass\ (\hmwkClassInstructor\ \hmwkClassTime): \hmwkTitle}
\rhead{\firstxmark}
\lfoot{\lastxmark}
\cfoot{\thepage}

\renewcommand\headrulewidth{0.4pt}
\renewcommand\footrulewidth{0.4pt}

\setlength\parindent{0pt}

%
% Create Problem Sections
%

\newcommand{\enterProblemHeader}[1]{
    \nobreak\extramarks{}{Problem \arabic{#1} continued on next page\ldots}\nobreak{}
    \nobreak\extramarks{Problem \arabic{#1} (continued)}{Problem \arabic{#1} continued on next page\ldots}\nobreak{}
}

\newcommand{\exitProblemHeader}[1]{
    \nobreak\extramarks{Problem \arabic{#1} (continued)}{Problem \arabic{#1} continued on next page\ldots}\nobreak{}
    \stepcounter{#1}
    \nobreak\extramarks{Problem \arabic{#1}}{}\nobreak{}
}

\setcounter{secnumdepth}{0}
\newcounter{partCounter}
\newcounter{homeworkProblemCounter}
\setcounter{homeworkProblemCounter}{1}
\nobreak\extramarks{Problem \arabic{homeworkProblemCounter}}{}\nobreak{}

%
% Homework Problem Environment
%
% This environment takes an optional argument. When given, it will adjust the
% problem counter. This is useful for when the problems given for your
% assignment aren't sequential. See the last 3 problems of this template for an
% example.
%
\newenvironment{homeworkProblem}[1][-1]{
    \ifnum#1>0
        \setcounter{homeworkProblemCounter}{#1}
    \fi
    \section{Problem \arabic{homeworkProblemCounter}}
    \setcounter{partCounter}{1}
    \enterProblemHeader{homeworkProblemCounter}
}{
    \exitProblemHeader{homeworkProblemCounter}
}

%
% Homework Details
%   - Title
%   - Due date
%   - Class
%   - Section/Time
%   - Instructor
%   - Author
%

\newcommand{\hmwkTitle}{Homework\ \#4}
\newcommand{\hmwkDueDate}{Mar 26, 2024}
\newcommand{\hmwkClass}{Real Analysis}
\newcommand{\hmwkClassTime}{Tuesday}
\newcommand{\hmwkClassInstructor}{Professor Yakun Xi}
\newcommand{\hmwkAuthorName}{\textbf{Shuang Hu}}

%
% Title Page
%

\title{
    \vspace{2in}
    \textmd{\textbf{\hmwkClass:\ \hmwkTitle}}\\
    \normalsize\vspace{0.1in}\small{Due\ on\ \hmwkDueDate\ at 10:00am}\\
    \vspace{0.1in}\large{\textit{\hmwkClassInstructor\ \hmwkClassTime}}
    \vspace{3in}
}

\author{\hmwkAuthorName}
\date{}

\renewcommand{\part}[1]{\textbf{\large Part \Alph{partCounter}}\stepcounter{partCounter}\\}

%
% Various Helper Commands
%

% Useful for algorithms
\newcommand{\alg}[1]{\textsc{\bfseries \footnotesize #1}}

% For derivatives
\newcommand{\deriv}[1]{\frac{\mathrm{d}}{\mathrm{d}x} (#1)}

% For partial derivatives
\newcommand{\pderiv}[2]{\frac{\partial}{\partial #1} (#2)}

% Integral dx
\newcommand{\dx}{\mathrm{d}x}

% Alias for the Solution section header
\newcommand{\solution}{\textbf{\large Solution}}
\newcommand{\norm}[1]{\|#1\|}
% Probability commands: Expectation, Variance, Covariance, Bias
\newcommand{\Var}{\mathrm{Var}}
\newcommand{\Cov}{\mathrm{Cov}}
\newcommand{\Bias}{\mathrm{Bias}}
\newcommand{\supp}{\text{supp}}
\newcommand{\Rn}{\mathbb{R}^{n}}
\newcommand{\dif}{\mathrm{d}}
\newcommand{\avg}[1]{\left\langle #1 \right\rangle}
\newcommand{\difFrac}[2]{\frac{\dif #1}{\dif #2}}
\newcommand{\pdfFrac}[2]{\frac{\partial #1}{\partial #2}}
\newcommand{\OFL}{\mathrm{OFL}}
\newcommand{\UFL}{\mathrm{UFL}}
\newcommand{\fl}{\mathrm{fl}}
\newcommand{\Eabs}{E_{\mathrm{abs}}}
\newcommand{\Erel}{E_{\mathrm{rel}}}
\newcommand{\DR}{\mathcal{D}_{\widetilde{LN}}^{n}}
\newcommand{\add}[2]{\sum_{#1=1}^{#2}}
\newcommand{\innerprod}[2]{\left<#1,#2\right>}
\newcommand{\Sc}{\mathcal{S}}
\newcommand{\F}{\mathcal{F}}
\newcommand{\E}{\mathcal{E}}
\newcommand{\A}{\mathcal{A}}
\newcommand{\cp}[2]{\cup_{#1=1}^{#2}}
\newcommand{\sm}[2]{\sum_{#1=1}^{#2}}
\newcommand{\M}{\mathcal{M}}
\newcommand{\Lc}{\mathcal{L}}
\newcommand\tbbint{{-\mkern -16mu\int}}
\newcommand\tbint{{\mathchar '26\mkern -14mu\int}}
\newcommand\dbbint{{-\mkern -19mu\int}}
\newcommand\dbint{{\mathchar '26\mkern -18mu\int}}
\newcommand\bint{
{\mathchoice{\dbint}{\tbint}{\tbint}{\tbint}}
}
\newcommand\bbint{
{\mathchoice{\dbbint}{\tbbint}{\tbbint}{\tbbint}}
}
\begin{document}
\maketitle
\pagebreak
\begin{homeworkProblem}
    If $E\in\Lc$ and $m(E)>0$, for any $\alpha<1$ 
    there is an open interval $I$ such that $m(E\cap I)>\alpha m(I)$.
\end{homeworkProblem}
\begin{proof}
    If $m(E)<\infty$, by Proposition 1.20, 
    $\forall\epsilon>0$, $\exists$ finite disjoint 
    open intervals $\{I_{j}\}_{j=1}^{n}$, set $U:=\cp{j}{n}I_{j}$, 
    it satisfies $m(E\triangle U)<\epsilon$.

    If $\exists 0<\alpha_{0}<1$, $\forall I_{j}$, 
    \begin{displaymath}
        m(E\cap I_{j})\le\alpha_{0}m(I_{j}),
    \end{displaymath}
    then:
    \begin{itemize}
        \item $m(E\cap U)\le\alpha_{0} m(U)$,
        \item $m(E\setminus U)\le m(E\triangle U)<\epsilon$.
    \end{itemize}
    Since $U$ be measurable:
    \begin{displaymath}
        m(E)=m(E\cap U)+m(E\setminus U)\le\alpha_{0} m(U)+\epsilon,
    \end{displaymath}
    and $m(E\triangle)U<\epsilon$ means $m(U)\le m(E)+\epsilon$. 
    It shows:
    \begin{displaymath}
        \forall\epsilon>0,\;m(E)\le\frac{\alpha_{0}+1}{1-\alpha_0}\epsilon.
    \end{displaymath}
    i.e. $m(E)=0$, contradicts!

    If $m(E)=\infty$, consider $E_{n}:=E\cap[-n,n]$, 
    as $\cp{n}{\infty}E_{n}=E$, $\exists N_{0}\in\mathbb{N}$ s.t. 
    $E_{N_0}\neq\emptyset$. Since $m(E_{N_0})<\infty$, $\forall\alpha<1$, 
    $\exists$ open interval $I$ s.t. 
    \begin{displaymath}
        m(E\cap I)\ge m(E_{N_0}\cap I)>\alpha m(I).
    \end{displaymath}
    Now we complete the proof.
\end{proof}
\begin{homeworkProblem}
    If $E\in\L$ and $m(E)>0$, the set $E-E:=\{x-y:x,y\in E\}$ 
    contains an interval centered at 0.
\end{homeworkProblem}
\begin{proof}
    By Problem 1, $\exists$ open interval $I$ s.t. 
    \begin{displaymath}
        m(E\cap I)>\frac{3}{4}m(I).
    \end{displaymath}
    If $|x|<\frac{m(I)}{2}$, we claim:
    \begin{equation}
        \label{eq:capeqphi}
        (E+x)\cap E=\emptyset.
    \end{equation}
    If $\exists$ $|x|<\frac{m(I)}{2}$ s.t. $(E+x)\cap E=\emptyset$, 
    assume $F=E\cap I$, it means $(F+x)\cap F=\emptyset$, i.e. 
    \begin{displaymath}
        2m(F)=m(F+x)+m(F)=m((F+x)\cup F).
    \end{displaymath} 
    Since $F\subset I$ and $|x|\le\frac{m(I)}{2}$, 
    $m((F+x)\cup F)\le\frac{3m(I)}{2}$, i.e. $m(F)\le \frac{3m(I)}{4}$, 
    contradict!
    
    So $\forall |x|<\frac{m(I)}{2}$, $(E+x)\cap E=\emptyset$, i.e. 
    $(-\frac{m(I)}{2},\frac{m(I)}{2})\subset E-E$.
\end{proof}
\begin{homeworkProblem}
    Suppose $\{\alpha_{j}\}_{1}^{\infty}\subset(0,1)$. 
    \begin{enumerate}
        \item $\prod_{1}^{\infty}(1-\alpha_j)>0$ iff 
        $\sm{j}{\infty}\alpha_{j}<\infty$.
        \item Given $\beta\in(0,1)$, exhibit a sequence $\{\alpha_j\}$ 
        s.t. $\prod_{1}^{\infty}(1-\alpha_{j})=\beta$.
    \end{enumerate}
\end{homeworkProblem}
\begin{proof}
    $(1)$ As $\alpha_{j}\in(0,1)$, $\prod_{1}^{\infty}(1-\alpha_{j})>0$ 
    means $\sm{j}{\infty}\log(1-\alpha_{j})$ converges. 
    
    On one hand, 
    if $\sm{j}{\infty}\log(1-\alpha_{j})$ converges to $c\in\mathbb{R}$, 
    since $\forall j$, $\log(1-\alpha_{j})<0$, it means: 
    $\forall N\in\mathbb{N}$
    \begin{displaymath}
        \sum_{i=1}^{N}\log(1-\alpha_{i})>c.
    \end{displaymath}
    As $\log(1-\alpha_{i})<-\alpha_{i}$, the equation above means 
    $\forall N\in\mathbb{N}$, $0<\sm{i}{N}\alpha_{i}<-c$. 
    For $\alpha_{i}>0$ is always true, 
    $\sm{i}{\infty}\alpha_{i}$ converges.

    On the other hand, if $\sm{j}{\infty}\alpha_{j}$ converges, 
    by Cauchy principle of convergence, $\forall\epsilon>0$, 
    $\exists N\in\mathbb{N}$ s.t. 
    \begin{displaymath}
        0<\sum_{i=N}^{\infty}\alpha_{i}<\epsilon_{0},
    \end{displaymath}
    where $\epsilon_{0}=\min\{\epsilon,\frac{1}{2}\}$. 
    Then:
    \begin{displaymath}
        \begin{array}{rl}
        |\sum_{i=N}^{\infty}\log(1-\alpha_{i})|&=
        -\sum_{i=N}^{\infty}\log(1-\alpha_{i})\\
        &=\sum_{i=N}^{\infty}\sum_{k=1}^{\infty}\frac{\alpha_{i}^{k}}{k}\\
        &=\sum_{k=1}^{\infty}\sum_{i=N}^{\infty}\frac{\alpha_{i}^{k}}{k}\\
        &\le\sm{k}{\infty}\frac{\epsilon_{0}^{k}}{k}
        <2\epsilon_{0}.
        \end{array}
    \end{displaymath}
    The second step follows from Taylor expansion, 
    the third from Tonelli's theorem, 
    the fourth step from 
    $\sum_{i=N}^{\infty}\alpha_{i}^{k}\le
    (\sum_{i=N}^{\infty}\alpha_{i})^{k}$.

    By Cauchy principle of convergence, 
    $\sum_{i=1}^{\infty}\log(1-\alpha_{i})$ converges. 
    So $\prod_{1}^{\infty}(1-\alpha_{j})>0$.

    $(2)$ It suffices to construct $\{\alpha_{i}\}$ s.t. 
    \begin{displaymath}
        \sm{i}{\infty}\log(1-\alpha_{i})=\log\beta.
    \end{displaymath}
    Choose $\log(1-\alpha_{i})=\frac{\beta}{2^{i}}$, i.e. 
    \begin{displaymath}
        \alpha_{i}=1-\exp(\frac{\beta}{2^{i}}),
    \end{displaymath}
    the equation is true.
\end{proof}
\begin{homeworkProblem}
    There exists a Borel set $A\subset[0,1]$ s.t. 
    $0<m(A\cap I)<m(I)$ for every 
    subinterval 
    $I$ of $[0,1]$.
\end{homeworkProblem}
\begin{proof}
    Construct the set $A$ as following steps:
    \begin{enumerate}
        \item Construct a Cantor-type set of measure $\beta$, 
        marked as $\mathcal{C}_{1}$.
        \item For each open intervals $I_{i}$ 
        eliminated on the construction process of first step, 
        construct Cantor-type sets $\mathcal{C}_{2i}$ of 
        measure $\beta m(I_{i})$ 
        satisfies $\mathcal{C}_{2i}\subset I_{i}$, 
        choose $\mathcal{C}_{2}:=\cp{i}{\infty}C_{2i}.$
        \item For the open intervals eliminated on the last step, 
        construct the Cantor-type sets as step2. 
        Then get a sequence of sets $\mathcal{C}_{i}$.
        \item $A:=\cp{i}{\infty}\mathcal{C}_{i}$.
    \end{enumerate}
    Each Cantor-type set is a Borel set, 
    so $A$ is a Borel set in $[0,1]$.
    $\forall I=(a,b)\subset[0,1]$, 
    $\exists I_{ij}\subset I$ which is the 
    eleminated open interval in the construction step $i$. 
    On one hand, 
    \begin{displaymath}
        m(A\cap I)>m(A\cap I_{ij})>0.    
    \end{displaymath}
    On the other hand, 
    \begin{displaymath}
        m(A\cap I)\le m(A)-m(I_{ij}\setminus(A\cap I_{ij}))
        \le m(A)-(1-\beta)m(I_{ij})<m(A).
    \end{displaymath}
    So $A$ is just the Borel set we need.
\end{proof}
\begin{homeworkProblem}
    If $f:X\rightarrow\overline{\mathbb{R}}$ and 
    $f^{-1}((r,\infty])\in\M$ for each $r\in\mathbb{Q}$, then 
    $f$ is measurable.
\end{homeworkProblem}
\begin{proof}
    By Proposition 2.1, it suffices to show:
    \begin{displaymath}
        \forall r\in\mathbb{R},\; 
        f^{-1}((r,\infty])\in\M.
    \end{displaymath}
    Since $\mathbb{Q}$ is dense in $\mathbb{R}$, 
    $\exists\{r_{n}\}\subset\mathbb{Q}$ satisfies:
    \begin{itemize}
        \item $\{r_{n}\}$ is decreasing.
        \item $\lim_{n\rightarrow\infty}r_{n}=r$.
    \end{itemize}
    Then:
    \begin{displaymath}
        \cp{n}{\infty}f^{-1}((r_{n},\infty])
        =f^{-1}(\cp{n}{\infty}(r_n,\infty])
        =f^{-1}((r,\infty]).
    \end{displaymath}
    Since $f^{-1}((r_n,\infty])\in\M$, 
    by the definition of $\sigma$-algebra, 
    $\cp{n}{\infty}f^{-1}((r_n,\infty])\in\M$. 
    i.e. $f^{-1}((r,\infty])\in\M$. 
\end{proof}
\begin{homeworkProblem}
    The supremum of an uncountable family of measurable 
    $\overline{\mathbb{R}}$-valued functions on $X$ can 
    fail to be measurable.
\end{homeworkProblem}
\begin{proof}
    Choose an unmeasurable set $\Sc\subset[0,1]$, 
    $\Sc$ must be an uncountable set (otherwise $m(\Sc)=0$).
    $\forall x\in\Sc$, we can construct 
    the characteristic function $\chi_{\{x\}}$, 
    it's clear that $\chi_{\{x\}}$ is measurable.

    Then construct the family 
    \begin{displaymath}
        \F:=\{\chi_{\{a\}}:a\in\Sc\},
    \end{displaymath}
    $\sup\F=\chi_{\Sc}$ isn't a measurable function, 
    for $\chi_{\Sc}^{-1}([0,1])=\Sc$ isn't measurable.
\end{proof}
\begin{homeworkProblem}
    If $f:\mathbb{R}\rightarrow\mathbb{R}$ is monotone, then $f$ 
    is Borel measurable.
\end{homeworkProblem}
\begin{proof}
    WLOG, assume $f$ is increasing. 
    It suffices to show: for a Borel set $\mathcal{B}$, 
    $f^{-1}(\mathcal{B})$ is also Borel. 

    By Proposition 2.1, we just need to show that 
    \begin{displaymath}
        \forall I=(-\infty,b)\subset\mathbb{R}, 
        f^{-1}(I)\in\mathcal{B}_{\mathbb{R}}.
    \end{displaymath}
    If $\{x:f(x)<b\}=\emptyset$, 
    $f^{-1}(I)=\emptyset\in\mathcal{B}_{\mathbb{R}}$.

    Otherwise, 
    set $\tau=\sup\{x:f(x)<b\}$, 
    as $f$ be increasing, 
    $\forall y<\tau$, $f(y)<b$. 

    If $\tau=\infty$, i.e. $f^{-1}(I)=\mathbb{R}$, it's a Borel set. 
    If $\tau<\infty$, $f^{-1}(I)=(-\infty,\tau)$, which is also a Borel set.
\end{proof}
\begin{homeworkProblem}
    Let $f:[0,1]\rightarrow[0,1]$ be the Cantor function, 
    and let $g(x)=f(x)+x$.
    \begin{enumerate}
        \item $g$ is a bijection from $[0,1]$ to $[0,2]$, 
        and $h=g^{-1}$ is continuous from $[0,2]$ to $[0,1]$.
        \item If $C$ is a Cantor set, $m(g(C))=1$.
        \item $g(C)$ contains a Lebesgue nonmeasurable set $A$. 
        Let $B=g^{-1}(A)$, $B$ is Lebesgue measurable but not Borel.
        \item There exist a Lebesgue measurable function $F$ and a 
        continuous function $G$ on $\mathbb{R}$ such that 
        $F\circ G$ is not Lebesgue measurable.
    \end{enumerate}
    \begin{proof}
        $(1)$ As $g(x)\ge f(x)\ge 0$, 
        $g(x)=f(x)+x\le 1+1=2$, 
        $g:[0,1]\rightarrow[0,2]$.

        By the definition of Cantor function, 
        $f(x)$ is increasing. 
        Since $\varphi(x)=x$ is strictly increasing, 
        $g(x)=f(x)+\varphi(x)$ is strictly increasing, 
        so $g(x)$ must be injective. 
        As $g(0)=0$ and $g(1)=2$, 
        $g(x)$ must be surjective.
        So $g$ is a bijection.

        Now we show $g^{-1}$ is left continuous, 
        the property of right continuous is in the same way. 
        
        Choose an increasing sequence $y_{n}$ 
        converges to $y$, and 
        $g(x_{n})=y_{n}$, $g(x)=y$, 
        it suffices to show $x_{n}\rightarrow x$.

        As $g$ be strictly increasing, 
        $x_{n}$ is strictly increasing and $x_{n}\le 1$. 
        So $x_{n}$ converges to $\tilde{x}\in[0,1]$. 
        By the continuity of $g$, 
        $$g(\tilde{x})=g(\lim_{n\rightarrow\infty}x_{n})
        =\lim_{n\rightarrow\infty}g(x_{n})=y.$$
    And $g$ is a bijection, it means that $\tilde{x}=x$.\qed 

    $(2)$ 
    On each subintervals $I$ in $[0,1]\setminus C$, 
    $f$ is a constant function, 
    and $f$ gets different constants on 
    different connective components in $[0,1]\setminus C$. 
    As $g(x)=x+f(x)$, 
    \begin{displaymath}
        m(g([0,1]\setminus C))
        =m([0,1]\setminus C)
        =1.
    \end{displaymath}
    As $g$ be a bijection, 
    \begin{displaymath}
        m(g(C)\cup g([0,1]\setminus C))
        =m(g(C))+m(g([0,1]\setminus C))
        =m(g[0,1])=2.
    \end{displaymath}
    So $m(g(C))=1$.\qed

    $(3)$ $A\subset g(C)$ means 
    $g^{-1}(A)\subset C$. 
    By the definition, $m(C)=0$. 
    As $B\subset C$ and $m$ be a complete measure, 
    $B$ is Lebesgue measurable with $m(B)=0$.

    As $g$ be a continuous map, if $B$ is a Borel set, 
    $f(B)=A$ is also a Borel set. 
    But $A$ isn't measurable, contradict! 
    So $B$ can't be a Borel set.\qed 

    $(4)$ Set $F:[0,1]\rightarrow [0,1]$ satisfies:
    \begin{itemize}
        \item $\forall x\in B$, $F(x)=1$.
        \item $\forall x\notin B$, $F(x)=0$. 
    \end{itemize}
    As $m(B)=0$, $F$ is Lebesgue measurable. 
    Set $G=g^{-1}$ which is continuous from $[0,2]$ to $[0,1]$, 
    then $F\circ G$ is a function from $[0,2]$ to $[0,1]$ 
    with $F$ Lebesgue measurable and $G$ continuous.

    However, $$
    (F\circ G)^{-1}([0,1])
    =G^{-1}(B)=g(B)=A
    $$
    and $A$ isn't measurable. 
    So $F\circ G$ isn't a Lebesgue measurable function.
    \end{proof}
\end{homeworkProblem}
\end{document}