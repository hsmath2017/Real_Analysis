\documentclass{article}

\usepackage{fancyhdr}
\usepackage{extramarks}
\usepackage{amsmath}
\usepackage{amsthm}
\newtheorem{lemma}{Lemma}
\usepackage{amsfonts}
\usepackage{tikz}
\usepackage[plain]{algorithm}
\usepackage{algpseudocode}

\usetikzlibrary{automata,positioning}

%
% Basic Document Settings
%

\topmargin=-0.45in
\evensidemargin=0in
\oddsidemargin=0in
\textwidth=6.5in
\textheight=9.0in
\headsep=0.25in

\linespread{1.1}

\pagestyle{fancy}
\lhead{\hmwkAuthorName}
\chead{\hmwkClass\ (\hmwkClassInstructor\ \hmwkClassTime): \hmwkTitle}
\rhead{\firstxmark}
\lfoot{\lastxmark}
\cfoot{\thepage}

\renewcommand\headrulewidth{0.4pt}
\renewcommand\footrulewidth{0.4pt}

\setlength\parindent{0pt}

%
% Create Problem Sections
%

\newcommand{\enterProblemHeader}[1]{
    \nobreak\extramarks{}{Problem \arabic{#1} continued on next page\ldots}\nobreak{}
    \nobreak\extramarks{Problem \arabic{#1} (continued)}{Problem \arabic{#1} continued on next page\ldots}\nobreak{}
}

\newcommand{\exitProblemHeader}[1]{
    \nobreak\extramarks{Problem \arabic{#1} (continued)}{Problem \arabic{#1} continued on next page\ldots}\nobreak{}
    \stepcounter{#1}
    \nobreak\extramarks{Problem \arabic{#1}}{}\nobreak{}
}

\setcounter{secnumdepth}{0}
\newcounter{partCounter}
\newcounter{homeworkProblemCounter}
\setcounter{homeworkProblemCounter}{1}
\nobreak\extramarks{Problem \arabic{homeworkProblemCounter}}{}\nobreak{}

%
% Homework Problem Environment
%
% This environment takes an optional argument. When given, it will adjust the
% problem counter. This is useful for when the problems given for your
% assignment aren't sequential. See the last 3 problems of this template for an
% example.
%
\newenvironment{homeworkProblem}[1][-1]{
    \ifnum#1>0
        \setcounter{homeworkProblemCounter}{#1}
    \fi
    \section{Problem \arabic{homeworkProblemCounter}}
    \setcounter{partCounter}{1}
    \enterProblemHeader{homeworkProblemCounter}
}{
    \exitProblemHeader{homeworkProblemCounter}
}

%
% Homework Details
%   - Title
%   - Due date
%   - Class
%   - Section/Time
%   - Instructor
%   - Author
%

\newcommand{\hmwkTitle}{Homework\ \#12}
\newcommand{\hmwkDueDate}{May 28, 2024}
\newcommand{\hmwkClass}{Real Analysis}
\newcommand{\hmwkClassTime}{Tuesday}
\newcommand{\hmwkClassInstructor}{Professor Yakun Xi}
\newcommand{\hmwkAuthorName}{\textbf{Shuang Hu}}

%
% Title Page
%

\title{
    \vspace{2in}
    \textmd{\textbf{\hmwkClass:\ \hmwkTitle}}\\
    \normalsize\vspace{0.1in}\small{Due\ on\ \hmwkDueDate\ at 10:00am}\\
    \vspace{0.1in}\large{\textit{\hmwkClassInstructor\ \hmwkClassTime}}
    \vspace{3in}
}

\author{\hmwkAuthorName}
\date{}

\renewcommand{\part}[1]{\textbf{\large Part \Alph{partCounter}}\stepcounter{partCounter}\\}

%
% Various Helper Commands
%

% Useful for algorithms
\newcommand{\alg}[1]{\textsc{\bfseries \footnotesize #1}}

% For derivatives
\newcommand{\deriv}[1]{\frac{\mathrm{d}}{\mathrm{d}x} (#1)}

% For partial derivatives
\newcommand{\pderiv}[2]{\frac{\partial}{\partial #1} (#2)}

% Integral dx
\newcommand{\dx}{\mathrm{d}x}

% Alias for the Solution section header
\newcommand{\solution}{\textbf{\large Solution}}
\newcommand{\norm}[1]{\|#1\|}
% Probability commands: Expectation, Variance, Covariance, Bias
\newcommand{\Var}{\mathrm{Var}}
\newcommand{\Cov}{\mathrm{Cov}}
\newcommand{\Bias}{\mathrm{Bias}}
\newcommand{\supp}{\text{supp}}
\newcommand{\Rn}{\mathbb{R}^{n}}
\newcommand{\dif}{\mathrm{d}}
\newcommand{\avg}[1]{\left\langle #1 \right\rangle}
\newcommand{\difFrac}[2]{\frac{\dif #1}{\dif #2}}
\newcommand{\pdfFrac}[2]{\frac{\partial #1}{\partial #2}}
\newcommand{\OFL}{\mathrm{OFL}}
\newcommand{\UFL}{\mathrm{UFL}}
\newcommand{\fl}{\mathrm{fl}}
\newcommand{\Eabs}{E_{\mathrm{abs}}}
\newcommand{\Erel}{E_{\mathrm{rel}}}
\newcommand{\DR}{\mathcal{D}_{\widetilde{LN}}^{n}}
\newcommand{\add}[2]{\sum_{#1=1}^{#2}}
\newcommand{\innerprod}[2]{\left<#1,#2\right>}
\newcommand{\Sc}{\mathcal{S}}
\newcommand{\F}{\mathcal{F}}
\newcommand{\E}{\mathcal{E}}
\newcommand{\A}{\mathcal{A}}
\newcommand{\cp}[2]{\cup_{#1=1}^{#2}}
\newcommand{\sm}[2]{\sum_{#1=1}^{#2}}
\newcommand{\M}{\mathcal{M}}
\newcommand{\Lc}{\mathcal{L}}
\newcommand\tbbint{{-\mkern -16mu\int}}
\newcommand\tbint{{\mathchar '26\mkern -14mu\int}}
\newcommand\dbbint{{-\mkern -19mu\int}}
\newcommand\dbint{{\mathchar '26\mkern -18mu\int}}
\newcommand\bint{
{\mathchoice{\dbint}{\tbint}{\tbint}{\tbint}}
}
\newcommand\bbint{
{\mathchoice{\dbbint}{\tbbint}{\tbbint}{\tbbint}}
}
\begin{document}
\maketitle
\pagebreak
\begin{homeworkProblem}
    When does equality hold in Minkowski's inequality?
\end{homeworkProblem}
\begin{proof}
    For $p=1$, 
    \begin{displaymath}
        \norm{f+g}_1=\int|f+g|\le\int(|f|+|g|)=\int|f|+\int|g|.
    \end{displaymath}
    $"="$ holds if and only if $|f+g|=|f|+|g|$ a.e., it means 
    \begin{displaymath}
        \text{for a.e. }z\in X,\;\overline{f(z)}g(z)\ge 0.
    \end{displaymath}
    
    For $1<p<\infty$, by the proof of Theorem 6.5, $"="$ means:
    \begin{itemize}
        \item $|f+g|=|f|+|g|$ a.e..
        \item \begin{displaymath}
            \begin{array}{rl}
                \alpha|f|^{p}&=\beta(|f+g|^{p-1})^{p'},\\
                \tilde{\alpha}|g|^{p}&=\tilde{\beta}(|f+g|^{p-1})^{p'},\\
            \end{array}
        \end{displaymath}
        where $p'=\frac{p}{p-1}$. So it means that 
        $g=\lambda f$ for $\lambda>0$ a.e..
    \end{itemize}

    For $p=\infty$, $\norm{f+g}_{\infty}=\norm{f}_{\infty}+\norm{g}_{\infty}$ means: 

    $\forall\epsilon>0$, $\exists$ $z\in X$ such that 
    \begin{itemize}
        \item $f(z)\ge\norm{f}_{\infty}-\epsilon$,
        \item $g(z)\ge\norm{g}_{\infty}-\epsilon$, 
        \item $f(z)=\lambda g(z)$ for $\lambda>0$.
    \end{itemize}
\end{proof}
\begin{homeworkProblem}
    Prove Theorem 6.8.
\end{homeworkProblem}
\begin{proof}
    $(a)$ By the definition:
    \begin{displaymath}
        \norm{fg}_1=\int|fg|\le\int \norm{g}_{\infty}|f|=\norm{g}_{\infty}\int|f|
        =\norm{f}_1\norm{g}_{\infty}.
    \end{displaymath}
    In the case $f\in L^{1}$ and $g\in L^{\infty}$, it means $\norm{g}_{\infty}<\infty$, 
    $\norm{f}_1<\infty$. Then $"="$ holds means $|fg|=|f|\norm{g}_{\infty}$ a.e.. 
    which means on the set $\{f\neq 0\}$, $|g|=\norm{g}_{\infty}$ a.e..

    $(b)$ By the definition, $\forall f\in L^{\infty}$, $\norm{f}_{\infty}\ge 0$. 

    $\norm{f}_{\infty}=0$ means $\forall\epsilon>0$, $\mu(\{x:|f(x)|>\epsilon\})=0$, 
    so $f=0$ a.e.. 

    For $\lambda\in\mathbb{C}$, $\mu(|f|>a)=0\Leftrightarrow \mu(|\lambda f|>|\lambda|a)=0$. 
    So $\norm{\lambda f}_{\infty}=\text{esssup}_{x\in X}|\lambda f|=|\lambda|\text{esssup}_{x\in X}|f|=|\lambda|\norm{f}_{\infty}$.

    Since $|f|\le\norm{f}_{\infty}$ a.e., $|g|\le\norm{g}_{\infty}$ a.e., 
    \begin{displaymath}
        |f+g|\le|f|+|g|\le\norm{f}_{\infty}+\norm{g}_{\infty}\text{ a.e.,}
    \end{displaymath}
    so $\norm{f+g}_{\infty}\le\norm{f}_{\infty}+\norm{g}_{\infty}$. 

    $(c)$ $"\Rightarrow"$: $\norm{f_n-f}_{\infty}\rightarrow 0$ means 
    $\forall\epsilon>0$, $\exists N$, $\forall n>N$, $|f_n-f|<\epsilon$ a.e.. 
    Then it means: $\forall n>N_{n}$, $|f_{n}-f|<\frac{1}{n}$ on $E_{n}$ 
    with $\mu(X\setminus E_{n})=0$. Choose $E:=\cp{n}{\infty}E_n$, 
    $\mu(X\setminus E)=0$. So $\forall n\in \mathbb{N}$, $\exists N_{n}$ such that 
    $\forall n>N_{n}$, $|f_{n}-f|<\frac{1}{n}$ on $E$, which means 
    $f_n\rightarrow f$ uniformly on $E$.

    $"\Leftarrow"$: $f_{n}\rightarrow f$ uniformly on $E$ means 
    $\forall\epsilon>0$, $\exists N$ such that $\forall x\in E$, $n>N$, 
    $|f_{n}(x)-f(x)|<\epsilon$. 
    Since $\mu(X\setminus E)=0$, we can see $\norm{f_n-f}_{\infty}\rightarrow 0$.

    $(d)$ Choose a Cauchy sequence $\{x_n\}$ on $L^{\infty}$, i.e. 
    \begin{displaymath}
        \forall\epsilon>0,\;\exists N,\;\forall m,n>N\quad \norm{x_{n}-x_{m}}<\epsilon.
    \end{displaymath}
    Assume $E_{n,m}(\epsilon):=\{t:|x_{n}(t)-x_{m}(t)|>\epsilon\}$, 
    then $\mu(E_{n,m}(\epsilon))=0$. If we choose 
    \begin{displaymath}
        F:=\cp{n}{\infty}\cup_{m,p\ge N(\frac{1}{n})}E_{m,p}(\frac{1}{n}),
    \end{displaymath}
    we have $\mu(F)=0$. So on $F^{c}$, $\forall n\in\mathbb{N}$, $\exists N$ such that 
    $\forall m,n>N$, $|x_{n}(t)-x_{m}(t)|<\frac{1}{n}$. 
    So if $t\in F^c$, $\{x_{n}(t)\}$ is a Cauchy sequence. 
    Now we set $x(t):=\lim_{n\rightarrow\infty}x_{n}(t)$ on $F^c$, $x(t):=0$ on $F$, 
    then $|x_{n}(t)-x_{m}(t)<\epsilon$ on $F^c$, which means $\{x_{n}(t)\}\rightarrow x(t)$ 
    uniformly on $F^c$. So $L^{\infty}$ is complete. 
    
    $(e)$ Choose $E:=\{x:|f(x)|\le\norm{f}_{\infty}\}$, then $\mu(X\setminus E)=0$. 
    By Theorem 2.10(b), $\exists$ a series of simple sunctions on $E$ such that 
    $x_{n}\rightarrow f$ uniformly on $E$. 
    Define $x_{n}(t)=0$ on $X\setminus E$, then by $(c)$, 
    $\norm{f-x_{n}}_{\infty}\Rightarrow 0$. 
    So simple functions are dense in $L^{\infty}$. 
\end{proof}
\begin{homeworkProblem}
    If $1\le p<r\le \infty$, $L^{p}\cap L^r$ is a Banach space with norm 
    $\norm{f}=\norm{f}_p+\norm{f}_r$, and if $p<q<r$, the inclusion map 
    $L^{p}\cap L^{r}\rightarrow L^{q}$ is continuous.
\end{homeworkProblem}
\begin{proof}
    Since $\norm{f}_p$, $\norm{f}_r$ are both norm, $\norm{f}$ is also a norm. 
    Assume $\{f_{n}\}$ is Cauchy related to $\norm{\cdot}$, i.e. 
    $\{f_{n}\}$ is Cauchy related to $\norm{\cdot}_{p},\norm{\cdot}_{q}$, 
    by completeness, $f_{n}\rightarrow f$ on $L^{p}\cap L^{q}$. 
    So $L^{p}\cap L^{q}$ is complete. 

    Since $\norm{f}_{q}\le\norm{f}_{p}^{\lambda}\norm{f}_{r}^{1-\lambda}$, 
    if $\norm{f_{n}}\rightarrow 0$, $\norm{f_{n}}_{p},\norm{f_{n}}_{r}\rightarrow 0$, 
    which means $\norm{f_{n}}_{q}\rightarrow 0$. So $i$ is continuous.
\end{proof}
\begin{homeworkProblem}
    If $1\le p<r\le\infty$, $L^{p}+L^{r}$ is a Banach space with norm 
    $\norm{f}=\inf\{\norm{g}_p+\norm{h}_{r}:f=g+h\}$, and if $p<q<r$, 
    $i:L^{q}\rightarrow L^{p}+L^{r}$ is continuous.
\end{homeworkProblem}
\begin{proof}
    NVS: By the definition, $\norm{f}\ge 0$ 
    and $\norm{f}=0\Leftrightarrow \norm{g}_p=\norm{h}_q=0\Leftrightarrow f=0$ a.e.. 

    \begin{displaymath}\norm{\lambda f}=\inf\{\norm{\lambda g}_{p}+\norm{\lambda h}_{r}\}=\inf\{|\lambda|(\norm{g}_{p}+\norm{h}_{r})\}=|\lambda|\norm{f}.
    \end{displaymath}
    \begin{displaymath}
    \norm{f_1+f_{2}}=\inf\{\norm{g_1+g_2}_{p}+\norm{h_1+h_2}_{r}\}
    \le\int\{\norm{g_1}_p+\norm{h_1}_{r}+\norm{g_2}_{p}+\norm{h_2}_{r}\}
    =\norm{f_{1}}+\norm{f_{2}}.
    \end{displaymath}
    Complete: if $\{f_{n}\}$ is Cauchy on $L^{p}+L^{r}$, 
    then $\{g_{n}\}$ is Cauchy on $L^{p}$ and 
    $\{h_{n}\}$ is Cauchy on $L^{r}$. 
    It means $g_{n}\rightarrow g$ on $L^{p}$ and $h_{n}\rightarrow h$ on $L^{r}$. 
    So $f_{n}\rightarrow f:=g+h$ on $L^{p}+L^{r}$. 

    If $\norm{f_{n}}_{q}\rightarrow 0$ on $L^{q}$, choose $E_{n}:=\{x:|f(x)|>1\}$, 
    then $g_{n}:=f_{n}\chi_{E_{n}}\in L^{p}$ and $h_{n}:=f_{n}(1-\chi_{E_n})\in L^{r}$, 
    and 
    \begin{displaymath}
        \norm{f_{n}}=\norm{g_{n}}_p+\norm{h_n}_{r}\le
        \norm{g_{n}}_{q}+\norm{h_{n}}_{q}
        \le 2\norm{f_{n}}_{q}\rightarrow 0.
    \end{displaymath}
    So $i:L^{q}\rightarrow L^{p}+L^{r}$ is continuous.
\end{proof}
\begin{homeworkProblem}
    If $f\in L^{p}\cap L^{\infty}$ for some $p<\infty$, 
    so that $f\in L^{q}$ for all $q>p$, 
    then $\norm{f}_{\infty}=\lim_{q\rightarrow\infty}\norm{f}_{q}$.
\end{homeworkProblem}
\end{document}